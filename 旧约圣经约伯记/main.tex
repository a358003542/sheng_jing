\documentclass[12pt,oneside]{book}

\usepackage{mybook}
\usepackage{mybookcover}


\title{旧约圣经约伯记}
\author{和合本}
\hypersetup{
    pdftitle={旧约圣经约伯记},
    pdfauthor={和合本},
    pdfcreator={Wander},
    pdfsubject={文学},
}


\begin{document}
\bookcover{book_cover.png}


\flypage[40pt][40pt]{你要以\\全部的心知\\全部的灵魂\\全部的头脑\\以及\\全部的气力\\来爱主你的神\\}



\frontmatter
\addchtoc{编者言}
\chapter*{编者言}
圣经卷帙浩繁,择其中部分单独成册。

原和合本为中英文对照版本,不便阅读,移除了英文。

原和合本中文每段前面都有 \verb+1:12+ 这样的考究标记,不便阅读,移除了。

原和合本有一些括号包围的内容,不便阅读,其中很多为翻译上的踌躇,将会对翻译作出取舍,若觉得有保留必要则将其转成脚注了。

本文翻译英文部分主要采用的是English Revised Version(ERV)版本,也会参考King James Bible(KJB)版本,一般差异不是太大,如果有需要权衡的地方,我可能会进一步询问希伯来原文,并考虑如何将希伯来原文的意思表达出来。中文部分主要参照的是现代标点和合本,并可能会对照其他翻译以及运用豆包AI翻译、词典等工具,从而做出一个自己更觉满意的翻译。原和合本翻译在意思上已经大差不差了,而个人觉得还有必要优化,主要是在中文这边的字词考究上。这些考究主要顾及:


\begin{itemize}
\item 更合乎现代中文用词习惯
\item 更具有中文人文底蕴
\item 在口语的地方会更倾向于口语化,而其他地方也会尽可能地通俗化,力图直白明了,想传达的意思别人一看就能领会,尤其是某些深层次的含义若能通过斟酌句子也能达到如上效果,那就更棒了。
\item 虽然不追求辞美,意达通顺即可,但有的时候偶然得到了自觉更美的字句,就不忍舍弃了。
\end{itemize}



\addchtoc{目录}
\setcounter{tocdepth}{2}    
\tableofcontents

\mainmatter
\chapter{第1章}
乌斯地有一个人名叫约伯。那人完全、正直,敬畏神,远离恶事。

他生了七个儿子,三个女儿。

他的家产有七千只羊、三千峰骆驼、五百对牛、五百头母驴,并有许多仆婢。这人在东方人中就为至大。

他的儿子们按着各自的日子,在自己家里设摆筵宴,又打发人去请他们的三个姐妹来,与他们一同吃喝。

筵宴的日子过了,约伯总要打发人去叫他们自洁。他大清早起来,按着他们众人的数目献燔祭。约伯如是说:“或许我的儿子们犯了罪,心里弃绝了神。”约伯常常这样行。

有一天,神的众子来侍立在耶和华面前,撒但也在其中。

耶和华问撒但:“你从哪里来?”撒但回答说:“我在地上往来行走,四处徘徊。”

耶和华问撒但:“你曾用心察看我的仆人约伯没有?地上再没有人像他那样完全、正直,敬畏神,远离恶事。”

撒但回答耶和华说:约伯敬畏神,岂是无故呢?

你岂不是四面圈上篱笆,围护他和他的家并他一切所有的吗?你赐福他手中的工作,他在地上的家产也增多了。

你且伸手毁他一切所有,他必当面弃绝你。

耶和华对撒但说:“凡他所有的都在你手中,只是不可伸手加害于他。”于是撒但从耶和华面前退去。

有一天,约伯的儿女们正在他们长兄的家里吃饭喝酒,

有报信的来见约伯说:牛正耕地、驴在旁边吃草之时,

示巴人突然袭击,将牲畜尽皆掳去,并用刀杀了仆人,唯有我一人逃脱,特来向你报信。

他还在说话的时候,又有一人前来报信:“神的火从天而降,将羊群和仆人都焚烧殆尽,唯有我一人逃脱,特来向你报信。”

他还在说话的时候,又有一人前来报信:“迦勒底人分作三队突袭,将骆驼全部掳走,并用刀杀了仆人,唯有我一人逃脱,特来向你报信。”

他还在说话的时候,又有一人前来报信:你的儿女们正在他们长兄的家里吃饭喝酒,

不料,有一阵大风从旷野刮来,重击房屋的四角,房屋倒塌在这些年轻人身上,他们都死了,唯有我一人逃脱,特来向你报信。

约伯便起来,撕裂外袍,剃光头发,伏在地上敬拜。

说:“我赤身出于母胎,也必赤身归回。耶和华赐下,耶和华收回,赞美主圣名\footnote{约伯此时所说应为日常用语,Blessed Be the Name of the Lord在宗教粤语歌中被翻译为赞美主圣名,更贴近此时场景。如果直译的话需要有被动结构的考虑,参考翻译为愿主名受称颂。但我建议少用主的被动的表达,主被动受美名当然是好事一件,但不是重点。重点在于主赐福,人有福了;或者人主动,当如是行。}。”

在这一切的事上约伯并未犯罪,也没有指责神行事愚妄。

\chapter{第2章}
又有一天,神的众子来侍立在耶和华面前,撒但也在其中。

耶和华问撒但:“你从哪里来?”撒但回答说:“我在地上往来行走,四处徘徊。”

耶和华问撒但:“你曾用心察看我的仆人约伯没有?地上再没有人像他那样完全、正直,敬畏神,远离恶事。你虽鼓动我与他作对,无缘无故地要毁灭他,他仍然持守自己的纯正。”

撒但回答耶和华说:以皮代皮,人情愿舍去一切所有的,保全性命。

你且伸手伤他的骨头和他的肉,他必当面弃绝你。

耶和华对撒但说:“他在你手中,但要存留他的性命。”

于是撒但从耶和华面前退去,击打约伯,使他从脚掌到头顶都长了毒疮。

约伯就坐在炉灰中,拿瓦片刮身体。

他的妻子对他说:“你仍然持守你的纯正吗?弃绝你的神,死掉算了\footnote{Curse God and die,这显然是他的妻子受撒但蛊惑说的话,Curse God and die即意味着撒但在与上帝的对问中胜出。}。”

约伯却对她说:“你说话像愚顽的妇人一样。哎,难道我们从神手里得福,就不能受祸吗?”在这一切的事上,约伯口中并没有犯罪。

约伯的三个朋友提幔人以利法、书亚人比勒达、拿玛人琐法,听说他遭遇的这一切灾祸,各人就从本处相约而来,来为他悲伤并安慰他。

他们远远地举目观看,竟认不出他来,就放声大哭。各人撕裂外袍,把尘土向天扬起来,落在自己的头上。

他们与约伯同坐在地上七天七夜,谁也不向他说一句话,因为他们见他的痛苦极其深重。


\chapter{第3章}
此后,约伯开口咒诅自己的生日,

说:

愿我出生的那日,还有我作为男孩受孕的那夜\footnote{此处the night应是特指的约伯被怀上的那一夜。},都灭亡吧。

愿那日变为黑暗,愿神不从上面眷顾它,愿光也不照于其上。

愿黑暗与死荫将其占据,愿乌云笼罩其上,愿一切使那日昏黑的事物都来恐吓它。

至于那夜,愿浓密的黑暗将其攫住,不让它在一年的日子里欢腾,也不让它列入月份的数目。

愿那夜没有生育,也没有欢乐的声音。

愿那些咒诅日子的人,就是那些随时准备唤醒利维坦[海怪]的人,都来咒诅那夜。

愿那夜黄昏的星辰暗淡无光,愿它寻觅光明却一无所获,也见不到清晨的曙光。

因它没有关闭我母亲的子宫,也没有将患难从我眼前隐藏。

为何我未从母腹中死去?为何我从母胎出来时没有断气?

为何有膝承接我?为何有奶哺养我?

不然,我早已躺下静息,早已沉睡,那时便得安宁——

与地上的君王和谋士同眠,他们曾为自己兴建荒陵和废冢;

或与拥有黄金的诸侯同眠,他们曾让居所堆满白银;

或如暗中流产的胎儿,未见天日的婴孩。

在那里,恶人止息搅扰,困乏人得享安息。

在那里,被囚之人同享安逸,听不见监工的呼喝之声。

在那里,大人小人皆在,奴仆已是自由之身。

为何要将光明赐给身处苦难之人,要将生命赐给心中愁苦之人?

他们切望死,却不得死;求死,胜于求隐藏的珍宝。

他们若能寻见坟墓,就会极度欢喜、满心快乐。

为何要将光明赐给那道路被遮蔽、又被神四面围困的人呢?

我在吃饭前就发出叹息,我的哀吼如水泉涌。

因我所恐惧的临到我身,我所惧怕的也来到我这里。

我不得安逸,不得平静,也不得安息,唯有患难降临。


\chapter{第4章}
提幔人以利法回答说:

若有人试着与你交谈,你还会悲伤吗?然而谁又能忍住不说话呢?

你曾教导过许多人,让他们软弱的手得以坚强。

你的话语扶持过跌倒的人,让他们虚弱的膝盖得以直立。

但如今灾祸临到你,你就昏厥;苦难触及你,你便烦乱。

你的倚靠不是在于你敬畏神吗?你的盼望不是在于你行事纯正吗?

请记住,我请求你,何曾有无辜者灭亡?哪里有正直人被剪除?

据我所见,耕罪孽者必收罪孽;种祸患者必受祸患。

因神的气息他们灭亡,因祂愤怒的狂风而被吞噬。

狮子的吼叫,猛狮的声音,尽都止息;幼狮的牙齿也都断掉。

老狮因缺乏猎物而死亡,母狮的幼崽也四处离散。

我暗暗得了默示,耳朵听到了细语。

在夜间异象的思绪中,世人都在沉睡的时候,

恐惧临到,战栗袭来,使我百骨乱战。

有灵从我面前经过,我身上的汗毛直竖。

那灵停住,我却不能辨其形状,有影像在我眼前。我在静默中听见有声音说:

凡人岂能比神更公义?人类岂能比他的造物主更纯洁?

主并不信任自己的仆人,且指责祂的天使行事愚昧,

何况那用泥做的骨肉?他们的根基立于尘土,飞蛾也能将其碾碎!

朝暮之间,就被毁灭,永远消亡,无人理会。

难道他们帐篷的绳索\footnote{帐篷的绳索对于游牧文明来说一个重要的隐喻,游牧文明以帐篷为家,帐篷的绳索很重要,代表着生命的支撑。}被拔起,不是因为自身败坏吗?他们死亡,且毫无智慧。


\chapter{第5章}
你且呼求,有谁回应你?诸圣者之中,你又转向哪一位呢?

烦恼害死愚妄人,嫉妒杀死愚笨人。

我曾见愚妄人根基稳固,但突然咒诅降临其处。

他的儿女远离安全,在城门口被欺压,却无人搭救。

他的收成被饥饿者吞食,连藏在荆棘里的也被夺走,罗网张开待其财物。

因为患难并非从尘土中生出,困苦也不是从地里冒出。

人生而困苦,如飞火向上。

至于我,我必寻求上帝,也必将我的案件交托于祂。

祂行大事不可测度,行奇事不可胜数。

降雨在地上,赐水于田里。

将卑微的人升高,将哀恸的高举至安稳。

祂使奸猾人诡计落空,故图谋之事双手难筹。

祂使智慧人陷入自己的机巧中,使乖僻人的计谋全盘落空。

他们白天遇见黑暗,正午摸索如在夜间。

祂救穷人脱离恶人口,也救他们脱离强人手。

这样,穷人便有了指望,恶人也闭口无言。

蒙神教训的人是有福的!所以你不可轻看全能者的惩戒。

因为祂使人疼,则必包扎;使人伤,则必亲手复原。

祂必在六次患难中搭救你;就是七次,邪恶也无法害你。

饥荒时,祂必救你免于饿死;战争时,祂必救你免于刀枪。

你必免于口舌之祸;灾殃临到,你也不必惧怕。

你必笑对灾害饥荒;地上野兽,你也不必惧怕。

因为你必于野外立石为盟\footnote{传统译法多取象征义,翻译为和石头结盟来形容人和自然的和谐,本译法侧重实指义,依据古代近东立石为盟的文化传统。},这样此间野兽将与你和睦相处。

你必知道你的帐篷平安;你查看你的羊圈,必一无所失。

你也必知道你的后裔繁多,你的后代如同地上的青草。

你必寿满而入坟墓,如同一捆麦子应时收藏。

看啊,我们刚琢磨的这个理,就是这样,你要听进去,知道这些对你有好处。


\chapter{第6章}
约伯回答说:

惟愿我的烦恼称一称,我的灾祸也一并放在天平上!

如今它们比大海边所有的沙子还重,因此我的言语多有轻率。

因全能者的箭已入我身,其中之毒我灵尽饮,上帝之惊骇更如大军列阵逼来。

野驴有草岂会嘶叫?公牛有料岂会低哞?

没味的食物不加盐能吃吗?鸡蛋清又有什么味道呢?

我的灵魂拒绝接触它们,它们之于我就像恶心的食物。

惟愿我能得偿所求,惟愿上帝赐我所望。

只请求上帝将我压碎,松开祂的手,将我剪除。

这样我仍可得安慰;是的,在这不停歇的苦痛中狂喜——因我未曾否认圣者的言语。

我有什么气力,等候至今?我的终点又在哪里,忍耐到现在?

难道我的体力坚如磐石?难道我的肉身硬如黄铜?

难道我毫无自助之力了吗?实效灵能\footnote{effectual working有更深的含义,有时可能需要翻译为有效灵工,请参看\href{https://www.my3bc.com/effectual-working-eph-415-16/}{这个网页}。}已全然离我而去了吗?

对于那行将昏厥之人,他的朋友应该施以怜悯,即便他已放弃了对全能者的敬畏。

我的兄弟们行事虚情假意,就像溪水,就像那溪谷河水尽东流。

这河水天冷结冰就发黑,其内有雪藏着。

天渐暖时,雪自不见。天再热些,河水消散。

顺溪而行的商队调转车头,向上深入荒野然后灭亡。

提玛的商队盼望着溪水,示巴的队伍等待着溪水。

他们因有盼望而蒙羞,到了那里,受尽困顿。

现在你们毫无援助,见了我的灾祸就畏惧害怕。

我岂说过“给我财物”?或者“从你们家产中送礼物给我”?

我岂说过“救我于敌人之手”?或者“赎我于欺压者的掌控”?

请教导我,我便沉默不语,以使我明白错在何处。

正直的言语多么有力啊!但你们的争辩到底在责备什么呢?

你们想责备我说的话?那些绝望之人的话语,不过像一阵风罢了。

你们竟抽签操纵孤儿,买卖朋友如同货物。

现在请你们看着我,我决不当面说谎。

宣判吧\footnote{return在法学上有宣判的意思,此句的设计是从法学语境再退回到日常语境,来表现约伯对沟通的包容态度。},我恳求诸位,不要让此处存有不公。回心转意吧,我的理由是正当的。

我的舌尖有不义之言?难道我的味蕾不能尝出恶事?



\chapter{第7章}
人在世上岂无争战?他的日子不像雇工的日子?

就像仆役切望着阴凉处,又像雇工盼望着他的薪水。

就连我生来也要这样虚度光阴,注定夜夜疲惫啊。

我躺下的时候便想:“明天什么时候起床?”可长夜漫漫,我辗转反侧,直至天明。

我的肉体被蛆虫包裹,被尘土覆盖。我的皮肤才刚愈合又重新溃烂。

我的日子转得比梭还快,无希望地飞逝而去。

求你纪念我,我的生命就像一阵风,我的眼睛再也看不见美好。

曾看过我的人,他眼中再也没有我。你曾注目于我,我却已不在。

就像云彩消散,人下阴间也不再上来。

他将不再回家,故土之人也不再识他。

因此我不缄吾口,我要说出我灵的痛苦,我要倾诉灵魂的苦楚。

我岂是海,海中的怪兽,你竟派守望者监视我?

我曾说:我的床必安慰我,我的榻必舒缓我的诉苦衷肠。

然而你竟用梦惊吓我,用异象恐吓我。

以致于我的灵魂宁要窒息而死,也不要我这身骨头。

我厌恶我的生命,不愿长活于世。留我独自一人吧,因我的日子尽是虚无。

人算什么,你竟以他为大,竟把你的心放在他身上,

你竟每日看他,每时试他?

你到何时才转眼不看我,才让我独留好缓口气呢?

哦你这世人的守望者,我若有罪,于你何妨?为何你要把我当作你的目标,以致于我自觉累赘呢?

你为何不赦免我的罪过,除掉我的罪孽?因为如今我就要躺卧在尘土中,你将尽力地寻我,但我已不在。


\chapter{第8章}
书亚人比勒达回答说:

你还要说这些话语到几时?你口中的空话为何如狂风吹个不停?

上帝难道会滥用裁决?全能者难道会败坏公义?

是不是你的儿女们对神犯了罪,祂才使他们受报应。

你若尽心请求上帝,向全能者祈祷。

你若清白且正直,如今祂必定为你而来,来使你合于公义之所兴旺发达。

你虽始于微末,必终于发达。

我恳求你为此询问,用心去问那古代,古代先祖们所探寻出来的事理。

(因我们一生不过须臾,对世事一无所知,因此在人世间的日子如梦幻泡影。)

他们岂不会教导你,将事理讲述给你,心里的话语倾诉给你?

香蒲离开泥泞怎能长大?芦荻没有水怎能生长\footnote{此处原和合本对蒲、芦、荻等字的采用非常精妙。最终否决了rush翻译为菖蒲的方案,菖草不太合适。}?

草尚青青之时,未被砍倒,却比其他百草都先枯萎。

凡忘记上帝的人,其人生轨迹也是如此。不信上帝的人,他们的希望注定破灭。

他的信心一碰就碎,他所信赖的是张蛛网。

他将依靠自己的房屋,房屋必倒塌,他急忙去扶,房屋却已不在。

太阳光照,他青青翠翠,嫩芽向前,越过花园边界。

他的根须在石头堆里摸索,专注于那片石头之地。

他若从所居之地被拔除,那地就不认他,说:“我从未见过你。”

看啊,这就是他人生之路的全部乐趣,而在这土地之上,则必有他人冒出。

你看,上帝必不丢弃完全人,也绝不支持作恶者的行径。

祂必让你满口笑声,欢呼满嘴。

恨恶你的必蒙羞,恶人的帐篷必归于无。



\chapter{第9章}
约伯回答说:

我知道真理就是这样,但人如何才能在上帝面前可称公义呢?

若祂愿意应对那人,那人所能应答怕不过千分之一。

祂心有大智慧,力量全能,有谁顽固和上帝作对,还能兴旺发达的?

祂移走群山,群山浑然不知,当祂发怒群山倾覆。

祂震动大地,大地离其本位,地之基柱都在颤抖。

祂命令太阳,太阳便不升起,群星也因此封藏。

祂独自铺开天宇,步行于海浪之上。

祂创造北斗、参星和昴星,并南方诸星座。

祂行大事不可测度,行奇事不可胜数。

祂从我旁边经过,我却看不见。祂经过继续向前,我也毫无觉察。

祂想要制服掌控的,谁能阻止?谁敢对祂说:“你做什么?”

上帝必不收回祂的怒气,拉哈伯的凶恶之徒们都俯伏在祂之下。

更不用说我了,我怎敢回答祂,还斟酌措辞同祂理论?

就算我称得上公义,也不敢回答祂,只是向那至上者祈求。

就算我呼求,祂也回应了我,我又怎敢相信祂垂听了我的声音。

因祂用风暴摧残我,无故倍增我的伤痛。

祂不让我稍作喘息,却让我满心怨苦。

若论全能者的力量,看啊,祂就在这里\footnote{此处there有意翻译为这里,以弱化空间距离感。}!若论审判,谁为我定个日期?

就算我本公义,我自己的口也要谴责我自己;就算我本完全,我自己的口也要证我任性。

就算我称得上完全,我也不想认识我自己,我鄙视我的人生\footnote{此句与苏格拉底 “认识你自己”“未经审视的人生不值得过” 两句名言形成反向呼应。约伯认为:凡人的完全是有限的,就算完全自我审视的人生也不值得过。}。

都是一样,因此我说,完全人和恶人,祂都要灭绝。

若无辜者忽然被灾祸杀害,祂就要在审判时取笑他。

世界被恶人玩弄,蒙蔽了那些法官们的双目。若这一切不是祂的本意,那又是谁的?

现如今我的日子跑得比邮差还快,它们飞逝而去,不见一丝美好。

又如那些快船急速驶过,又如那雄鹰猛扑猎物。

我若说:我愿忘却我的哀怨,我愿放下我的愁容,做一个喜乐的人。

我惧怕我那所有的悲痛,我知道你必不以我为无辜。

我必被定为有罪,那我又何苦徒劳?

如我用雪水洗手,使我的手从未如此洁净。

你却把我扔进泥沟里,使我的衣服都厌恶我自己。

因祂不是如我一般的凡人,我怎么能回答祂,我们又怎么能一起参与审判。

我们之间没有调解人,能按双手在我们身上。

愿祂权杖离开我身,使我不再惧怕上帝之惊骇。

那样我就开口说话,也不恐惧祂,因我本性并非如此。



\chapter{第10章}
我的人生让灵魂疲惫不堪,我要尽情地诉苦埋怨,倾诉灵魂的苦楚。

我要对上帝说:请不要定我有罪,请指示我,为何你要为难我。

你竟压迫、轻视你亲手所作的,却光照恶人的计谋,这对你有何益处?

难道你的眼是血肉所造?难道你所见也如世人所看?

难道你的日子也如同凡人一般?难道你的年岁也像世人一样?

为何你如此追究我的罪孽,搜寻我的罪行?

其实,你知道我没有罪恶,并没有能救我脱离你手的。

你的手创造我,造就我的四肢百体,你还要毁灭我。

求你记念制造我如抟泥一般,你还要使我归于麈土吗。

你不是倒出我来好像奶,使我凝结如同奶饼吗。

你以皮和肉为衣给我穿上,用骨与筋把我全体联络。

你将生命和慈爱赐给我,你也眷顾保全我的心灵。

然而,你待我的这些事早已藏在你心里,我知道你久有此意。

我若犯罪,你就察看我,并不赦免我的罪孽。

我若行恶,便有了祸。我若为义,也不敢抬头。正是满心羞愧,眼见我的苦情。

我若昂首自得,你就追捕我如狮子,又在我身上显出奇能。

你重立见证攻击我,向我加增恼怒,如军兵更换着攻击我。

你为何使我出母胎呢。不如我当时气绝,无人得见我。

这样,就如没有我一般,一出母胎就被送入坟墓。

我的日子不是甚少吗。求你停手宽容我,叫我在往而不返之先,

就是往黑暗,和死荫之地以先,可以稍得畅快。

那地甚是幽暗,是死荫混沌之地。那里的光好像幽暗。



\chapter{第11章}
拿玛人琐法回答说,

这许多的言语岂不该回答吗。多嘴多舌的人岂可称为义吗。

你夸大的话,岂能使人不作声吗ⅶ你戏笑的时候,岂没有人叫你害羞吗。

你说,我的道理纯全,我在你眼前洁净。

惟愿神说话,愿他开口攻击你。

并将智慧的奥秘指示你,他有诸般的智识。所以当知道神追讨你,比你罪孽该得的还少。

你考察,就能测透神吗。你岂能尽情测透全能者吗。

他的智慧高于天,你还能做什么。深于阴间,你还能知道什么。

其量比地长,比海宽。

他若经过,将人拘禁,招人受审,谁能阻挡他呢。

他本知道虚妄的人。人的罪孽,他虽不留意,还是无所不见。

空虚的人却毫无知识。人生在世好像野驴的驹子。

你若将心安正,又向主举手。

你手里若有罪孽,就当远远地除掉,也不容非义住在你帐棚之中。

那时,你必仰起脸来毫无斑点。你也必坚固,无所惧怕。

你必忘记你的苦楚,就是想起也如流过去的水一样。

你在世的日子要比正午更明,虽有黑暗仍像早晨。

你因有指望就必稳固,也必四围巡查,坦然安息。

你躺卧,无人惊吓,且有许多人向你求恩。

但恶人的眼目必要失明。他们无路可逃。他们的指望就是气绝。



\chapter{第12章}
约伯回答说,

你们真是子民哪,你们死亡,智慧也就灭没了。

但我也有聪明,与你们一样,并非不及你们。你们所说的,谁不知道呢。

我这求告神,蒙他应允的人,竟成了朋友所讥笑的。公义完全人,竟受了人的讥笑。

安逸的人,心里藐视灾祸。这灾祸常常等待滑脚的人。

强盗的帐棚兴旺,惹神的人稳固,神多将财物送到他们手中。

你且问走兽,走兽必指教你。又问空中的飞鸟,飞鸟必告诉你。

或与地说话,地必指教你。海中的鱼也必向你说明。

看这一切,谁不知道是耶和华的手做成的呢。

凡活物的生命,和人类的气息,都在他手中。

耳朵岂不试验言语,正如上膛尝食物吗。

年老的有智慧,寿高的有知识。

在神有智慧和能力,他有谋略和知识。

他拆毁的,就不能再建造。他捆住人,便不得开释。

他把水留住,水便枯乾。他再发出水来,水就翻地。

在他有能力和智慧。被诱惑的,与诱惑人的,都是属他。

他把谋士剥衣掳去,又使审判官变成愚人。

他放松君王的绑,又用带子捆他们的腰。

他把祭司剥衣掳去,又使有能的人倾败。

他废去忠信人的讲论,又夺去老人的聪明。

他使君王蒙羞被辱,放松有力之人的腰带。

他将深奥的事从黑暗中彰显,使死荫显为光明。

他使邦国兴旺而又毁灭,他使邦国开广而又掳去。

他将地上民中首领的聪明夺去,使他们在荒废无路之地漂流。

他们无光,在黑暗中摸索,又使他们东倒西歪,像醉酒的人一样。



\chapter{第13章}
这一切,我眼都见过。我耳都听过,而且明白。

你们所知道的,我也知道,并非不及你们。

我真要对全能者说话。我愿与神理论。

你们是编造谎言的,都是无用的医生。

惟愿你们全然不作声。这就算为你们的智慧。

请你们听我的辩论,留心听我口中的分诉。

你们要为神说不义的话吗,为他说诡诈的言语吗。

你们要为神徇情吗,要为他争论吗。

他查出你们来,这岂是好吗。人欺哄人,你们也要照样欺哄他吗。

你们若暗中徇情,他必要责备你们。

他的尊荣岂不叫你们惧怕吗。他的惊吓岂不临到你们吗。

你们以为可记念的箴言是炉灰的箴言。你们以为可靠的坚垒是淤泥的坚垒。

你们不要作声,任凭我吧。让我说话,无论如何我都承当。

我何必把我的肉挂在牙上,将我的命放在手中。

他必杀我。我虽无指望,然而我在他面前还要辩明我所行的。

这要成为我的拯救,因为不虔诚的人,不得到他面前。

你们要细听我的言语,使我所辩论的入你们的耳中。

我已陈明我的案,知道自己有义。

有谁与我争论,我就情愿缄默不言,气绝而亡。

惟有两件不要向我施行,我就不躲开你的面。

就是把你的手缩回,远离我身。又不使你的惊惶威吓我。

这样,你呼叫,我就回答。或是让我说话,你回答我。

我的罪孽和罪过有多少呢。求你叫我知道我的过犯与罪愆。

你为何掩面,拿我当仇敌呢。

你要惊动被风吹的叶子吗。要追赶枯乾的碎秸吗。

你按罪状刑罚我,又使我担当幼年的罪孽。

也把我的脚上了木狗,并窥察我一切的道路,为我的脚掌划定界限。

我已经像灭绝的烂物,像虫蛀的衣裳。



\chapter{第14章}
人为妇人所生,日子短少,多有患难。

出来如花,又被割下。飞去如影,不能存留。

这样的人你岂睁眼看他吗。又叫我来受审吗。

谁能使洁净之物出于污秽之中呢。无论谁也不能。

人的日子既然限定,他的月数在你那里,你也派定他的界限,使他不能越过。

便求你转眼不看他,使他得歇息。直等他像雇工人完毕他的日子。

树若被砍下,还可指望发芽,嫩枝生长不息。

其根虽然衰老在地里,干也死在土中。

及至得了水气,还要发芽,又长枝条,像新栽的树一样。

但人死亡而消灭。他气绝,竟在何处呢。

海中的水绝尽,江河消散乾涸。

人也是如此,躺下不再起来。等到天没有了,仍不得复醒,也不得从睡中唤醒。

惟愿你把我藏在阴间,存于隐密处,等你的忿怒过去。愿你为我定了日期,记念我。

人若死了岂能再活呢。我只要在我一切争战的日子,等我被释放(被释放或作改变)的时候来到。

你呼叫,我便回答。你手所做的,你必羡慕。

但如今你数点我的脚步,岂不窥察我的罪过吗。

我的过犯被你封在囊中,也缝严了我的罪孽。

山崩变为无有。磐石挪开原处。

水流消磨石头,所流溢的洗去地上的尘土。你也照样灭绝人的指望。

你攻击人常常得胜,使他去世。你改变他的容貌,叫他往而不回。

他儿子得尊荣,他也不知道。降为卑,他也不觉得。

但知身上疼痛,心中悲哀。


\chapter{第15章}
提幔人以利法回答说,

智慧人岂可用虚空的知识回答,用东风充满肚腹呢。

他岂可用无益的话,和无济于事的言语,理论呢。

你是废弃敬畏的意,在神面前阻止敬虔的心。

你的罪孽指教你的口。你选用诡诈人的舌头。

你自己的口定你有罪,并非是我。你自己的嘴,见证你的不是。

你岂是头一个被生的人吗。你受造在诸山之先吗。

你曾听见神的密旨吗。你还将智慧独自得尽吗。

你知道什么是我们不知道的呢。你明白什么是我们不明白的呢。

我们这里有白发的,和年纪老迈的,比你父亲还老。

神用温和的话安慰你,你以为太小吗。

你的心为何将你逼去。你的眼为何冒出火星。

使你的灵反对神,也任你的口发这言语。

人是什么,竟算为洁净呢。妇人所生的是什么,竟算为义呢。

神不信靠他的众圣者。在他眼前,天也不洁净。

何况那污秽可憎,喝罪孽如水的世人呢。

我指示你,你要听。我要述说所看见的。

就是智慧人从列祖所受,传说而不隐瞒的。

(这地惟独赐给他们,并没有外人从他们中间经过)。

恶人一生之日劬劳痛苦,强暴人一生的年数也是如此。

惊吓的声音常在他耳中。在平安时,抢夺的必临到他那里。

他不信自己能从黑暗中转回,他被刀剑等候。

他漂流在外求食,说,那里有食物呢。他知道黑暗的日子,在他手边预备好了。

急难困苦叫他害怕,而且胜了他,好像君王预备上阵一样。

他伸手攻击神,以骄傲攻击全能者。

挺着颈项,用盾牌的厚凸面,向全能者直闯。

是因他的脸蒙上脂油,腰积成肥肉。

他曾住在荒凉城邑,无人居住,将成乱堆的房屋。

他不得富足,财物不得常存,产业在地上也不加增。

他不得出离黑暗。火焰要将他的枝子烧乾。因神口中的气,他要灭亡(灭亡原文作走去)。

他不用倚靠虚假欺哄自己,因虚假必成为他的报应。

他的日期未到之先,这事必成就。他的枝子不得青绿。

他必像葡萄树的葡萄,未熟而落。又像橄榄树的花,一开而谢。

原来不敬虔之辈必无生育。受贿赂之人的帐棚必被火烧。

他们所怀的是毒害,所生的是罪孽,心里所豫备的是诡诈。



\chapter{第16章}
约伯回答说,

这样的话我听了许多。你们安慰人,反叫人愁烦。

虚空的言语有穷尽吗。有什么话惹动你回答呢。

我也能说你们那样的话。你们若处在我的境遇,我也会联络言语攻击你们,又能向你们摇头。

但我必用口坚固你们,用嘴消解你们的忧愁。

我虽说话,忧愁仍不得消解。我虽停住不说,忧愁就离开我吗。

但现在神使我困倦,使亲友远离我。

又抓住我,作见证攻击我。我身体的枯瘦,也当面见证我的不是。

主发怒撕裂我,逼迫我,向我切齿。我的敌人怒目看我。

他们向我开口,打我的脸羞辱我,聚会攻击我。

神把我交给不敬虔的人,把我扔到恶人的手中。

我素来安逸,他折断我,掐住我的颈项,把我摔碎。又立我为他的箭靶子。

他的弓箭手四面围绕我。他破裂我的肺腑,并不留情,把我的胆倾倒在地上,

将我破裂又破裂,如同勇士向我直闯。

我缝麻布在我皮肤上,把我的角放在尘土中。

我的脸因哭泣发紫,在我的眼皮上有死荫。

我的手中却无强暴。我的祈祷也是清洁。

地阿,不要遮盖我的血。不要阻挡我的哀求。

现今,在天有我的见证,在上有我的中保。

我的朋友讥诮我,我却向神眼泪汪汪。

愿人得与神辩白,如同人与朋友辩白一样。

因为再过几年,我必走那往而不返之路。


\chapter{第17章}
我的心灵消耗,我的日子灭尽。坟墓为我预备好了。

真有戏笑我的在我这里,我眼常见他们惹动我。

愿主拿凭据给我,自己为我作保。在你以外谁肯与我击掌呢。

因你使他们心不明理,所以你必不高举他们。

控告他的朋友,以朋友为可抢夺的,连他儿女的眼睛也要失明。

神使我作了民中的笑谈。他们也吐唾沫在我脸上。

我的眼睛因忧愁昏花。我的百体好像影儿。

正直人因此必惊奇。无辜的人,要兴起攻击不敬虔之辈。

然而,义人要持守所行的道。手洁的人要力上加力。

至于你们众人,可以再来辩论吧。你们中间,我找不着一个智慧人。

我的日子已经过了。我的谋算,我心所想望的已经断绝。

他们以黑夜为白昼,说,亮光近乎黑暗。

我若盼望阴间为我的房屋,若下榻在黑暗中,

若对朽坏说,你是我的父。对虫说,你是我的母亲姊妹。

这样,我的指望在那里呢。我所指望的谁能看见呢。

等到安息在尘土中,这指望必下到阴间的门闩那里了。


\chapter{第18章}
书亚人比勒达回答说,

你寻索言语要到几时呢。你可以揣摩思想,然后我们就说话。

我们为何算为畜生,在你眼中看作污秽呢。

你这恼怒将自己撕裂的,难道大地为你见弃,磐石挪开原处吗。

恶人的亮光必要熄灭。他的火焰必不照耀。

他帐棚中的亮光要变为黑暗。他以上的灯也必熄灭。

他坚强的脚步必见狭窄。自己的计谋必将他绊倒。

因为他被自己的脚陷入网中,走在缠人的网罗上。

圈套必抓住他的脚跟。机关必擒获他。

活扣为他藏在土内。羁绊为他藏在路上。

四面的惊吓要使他害怕,并且追赶他的脚跟。

他的力量必因饥饿衰败,祸患要在他旁边等候。

他本身的肢体要被吞吃,死亡的长子要吞吃他的肢体。

他要从所倚靠的帐棚被拔出来,带到惊吓的王那里。

不属他的必住在他的帐棚里。硫磺必撒在他所住之处。

下边,他的根本要枯乾。上边,他的枝子要剪除。

他的记念在地上必然灭亡。他的名字在街上也不存留。

他必从光明中被撵到黑暗里,必被赶出世界。

在本民中必无子无孙。在寄居之地也无一人存留。

以后来的要惊奇他的日子,好像以前去的受了惊骇。

不义之人的住处总是这样。此乃不认识神之人的地步。


\chapter{第19章}
约伯回答说,

你们搅扰我的心,用言语压碎我,要到几时呢。ⅶ

你们这十次羞辱我。你们苦待我也不以为耻。

果真我有错,这错乃是在我。

你们果然要向我夸大,以我的羞辱为证指责我。

就该知道是神倾覆我,用网罗围绕我。

我因委曲呼叫,却不蒙应允。我呼求,却不得公断。

神用篱笆拦住我的道路,使我不得经过。又使我的路径黑暗。

他剥去我的荣光,摘去我头上的冠冕。

他在四围攻击我,我便归于死亡,将我的指望如树拔出来。

他的忿怒向我发作,以我为敌人。

他的军旅一齐上来,修筑战路攻击我。在我帐棚的四围安营。

他把我的弟兄隔在远处,使我所认识的,全然与我生疏。

我的亲戚与我断绝。我的密友都忘记我。

在我家寄居的,和我的使女都以我为外人。我在他们眼中看为外邦人。

我呼唤仆人,虽用口求他,他还是不回答。

我口的气味,我妻子厌恶。我的恳求,我同胞也憎嫌。

连小孩子也藐视我。我若起来,他们都嘲笑我。

我的密友都憎恶我。我平日所爱的人向我翻脸。

我的皮肉紧贴骨头。我只剩牙皮逃脱了。

我朋友阿,可怜我。可怜我。因为神的手攻击我。

你们为什么彷佛神逼迫我,吃我的肉还以为不足呢。

惟愿我的言语现在写上,都记录在书上。

用铁笔镌刻,用铅灌在磐石上,直存到永远。

我知道我的救赎主活着,末了必站立在地上。

我这皮肉灭绝之后,我必在肉体之外得见神。

我自己要见他,亲眼要看他,并不像外人。我的心肠在我里面消灭了。

你们若说,我们逼迫他要何等地重呢。惹事的根乃在乎他。

你们就当惧怕刀剑。因为忿怒惹动刀剑的刑罚,使你们知道有报应(原文作审判)。


\chapter{第20章}
拿玛人琐法回答说,

我心中急躁,所以我的思念叫我回答。

我已听见那羞辱我,责备我的话。我的悟性叫我回答。

你岂不知亘古以来,自从人生在地。

恶人夸胜是暂时的,不敬虔人的喜乐,不过转眼之间吗。

他的尊荣虽达到天上,头虽顶到云中,

他终必灭亡,像自己的粪一样。素来见他的人要说,他在那里呢。

他必飞去如梦,不再寻见,速被赶去,如夜间的异象。

亲眼见过他的,必不再见他。他的本处也再见不着他。

他的儿女要求穷人的恩。他的手要赔还不义之财。

他的骨头虽然有青年之力,却要和他一同躺卧在尘土中。

他口内虽以恶为甘甜,藏在舌头底下。

爱恋不舍,含在口中。

他的食物在肚里却要化为酸,在他里面成为虺蛇的恶毒。

他吞了财宝,还要吐出。神要从他腹中掏出来。

他必吸饮虺蛇的毒气。蝮蛇的舌头也必杀他。

流奶与蜜之河,他不得再见。

他劳碌得来的要赔还,不得享用(原文作吞下),不能照所得的财货欢乐。

他欺压穷人,且又离弃。强取非自己所盖的房屋(或作强取房屋不得再建造)。

他因贪而无厌,所喜悦的连一样也不能保守。

其馀的没有一样他不吞灭,所以他的福乐不能长久。

他在满足有馀的时候,必到狭窄的地步。凡受苦楚的人,都必加手在他身上。

他正要充满肚腹的时候,神必将猛烈的忿怒,降在他身上。正在他吃饭的时候,要将这忿怒像雨降在他身上。

他要躲避铁器。铜弓的箭要将他射透。

他把箭一抽,就从他身上出来。发光的箭头从他胆中出来,有惊惶临在他身上。

他的财宝归于黑暗。人所不吹的火要把他烧灭,要把他帐棚中所剩下的烧毁。

天要显明他的罪孽,地要兴起攻击他。

他的家产必然过去。神发怒的日子,他的货物都要消灭。

这是恶人从神所得的分,是神为他所定的产业。


\chapter{第21章}
约伯回答说,

你们要细听我的言语,就算是你们安慰我。

请宽容我,我又要说话。说了以后,任凭你们嗤笑吧。

我岂是向人诉冤,为何不焦急呢。

你们要看着我而惊奇,用手捂口。

我每逢思想,心就惊惶,浑身战兢。

恶人为何存活,享大寿数,势力强盛呢。

他们眼见儿孙,和他们一同坚立。

他们的家宅平安无惧。神的杖也不加在他们身上。

他们的公牛孳生而不断绝。母牛下犊而不掉胎。

他们打发小孩子出去,多如羊群。他们的儿女踊跃跳舞。

他们随着琴鼓歌唱,又因箫声欢喜。

他们度日诸事亨通,转眼下入阴间。

他们对神说,离开我们吧。我们不愿晓得你的道。

全能者是谁,我们何必事奉他呢。求告他有什么益处呢。

看哪,他们亨通不在乎自己。恶人所谋定的离我好远。

恶人的灯何尝熄灭。患难何尝临到他们呢。神何尝发怒,向他们分散灾祸呢。

他们何尝像风前的碎秸,如暴风刮去的糠秕呢。

你们说,神为恶人的儿女积蓄罪孽。我说,不如本人受报,好使他亲自知道。

愿他亲眼看见自己败亡,亲自饮全能者的忿怒。

他的岁月既尽,他还顾他本家吗。

神既审判那在高位的,谁能将知识教训他呢。

有人至死身体强壮,尽得平靖安逸。

他的奶桶充满,他的骨髓滋润。

有人至死心中痛苦,终身未尝福乐的滋味。

他们一样躺卧在尘土中,都被虫子遮盖。

我知道你们的意思,并诬害我的计谋。

你们说,霸者的房屋在那里。恶人住过的帐棚在那里。

你们岂没有询问过路的人吗。不知道他们所引的证据吗。

就是恶人在祸患的日子得存留,在发怒的日子得逃脱。

他所行的,有谁当面给他说明。他所做的,有谁报应他呢。

然而他要被抬到茔地,并有人看守坟墓。

他要以谷中的土块为甘甜,在他以先去的无数,在他以后去的更多。

你们对答的话中既都错谬,怎吗徒然安慰我呢。


\chapter{第22章}
提幔人以利法回答说,

人岂能使神有益呢。智慧人但能有益于己。

你为人公义,岂叫全能者喜悦呢。你行为完全,岂能使他得利呢。

岂是因你敬畏他,就责备你,审判你吗。

你的罪恶岂不是大吗。你的罪孽也没有穷尽。

因你无故强取弟兄的物为当头,剥去贫寒人的衣服。

困乏的人,你没有给他水喝。饥饿的人,你没有给他食物。

有能力的人就得地土。尊贵的人也住在其中。

你打发寡妇空手回去,折断孤儿的膀臂。

因此,有网罗环绕你,有恐惧忽然使你惊惶。

或有黑暗蒙蔽你,并有洪水淹没你。

神岂不是在高天吗。你看星宿何其高呢。

你说,神知道什么。他岂能看透幽暗施行审判呢。

密云将他遮盖,使他不能看见。他周游穹苍。

你要依从上古的道吗。这道是恶人所行的。

他们未到死期,忽然除灭。根基毁坏,好像被江河冲去。

他们向神说,离开我们吧。又说,全能者能把我们怎吗样呢。

那知,神以美物充满他们的房屋。但恶人所谋定的离我好远。

义人看见他们的结局就欢喜。无辜的人嗤笑他们。

说,那起来攻击我们的果然被剪除,其馀的都被火烧灭。

你若认识神,就得平安。福气也必临到你。

你当领受他口中的教训,将他的言语存在心里。

你若向全能者,从你帐棚中远除不义,就必得建立。

要将你的珍宝丢在尘土里,将俄斐的黄金丢在溪河石头之间。

全能者就必为你的珍宝,作你的宝银。

你就要以全能者为喜乐,向神仰起脸来。

你要祷告他,他就听你。你也要还你的愿。

你定意要作何事,必然给你成就。亮光也必照耀你的路。

人使你降卑,你仍可说,必得高升。谦卑的人神必拯救。

人非无辜,神且要搭救他。他因你手中清洁,必蒙拯救。


\chapter{第23章}
约伯回答说,

如今我的哀告还算为悖逆。我的责罚比我的唉哼还重。

惟愿我能知道在那里可以寻见神,能到他的台前。

我就在他面前将我的案件陈明,满口辩白。

我必知道他回答我的言语,明白他向我所说的话。

他岂用大能与我争辩吗。必不这样,他必理会我。

在他那里,正直人可以与他辩论。这样,我必永远脱离那审判我的。

只是,我往前行,他不在那里,往后退,也不能见他。

他在左边行事,我却不能看见,在右边隐藏,我也不能见他。

然而他知道我所行的路。他试炼我之后,我必如精金。

我脚追随他的步履。我谨守他的道,并不偏离。

他嘴唇的命令,我未曾背弃。我看重他口中的言语,过于我需用的饮食。

只是他心志已定,谁能使他转意呢。他心里所愿的,就行出来。

他向我所定的,就必做成。这类的事他还有许多。

所以我在他面前惊惶,我思念这事,便惧怕他。

神使我丧胆,全能者使我惊惶。

我的恐惧,不是因为黑暗,也不是因为幽暗蒙蔽我的脸。


\chapter{第24章}
全能者既定期罚恶,为何不使认识他的人看见那日子呢。

有人挪移地界,抢夺群畜而牧养。

他们拉去孤儿的驴,强取寡妇的牛为当头。

他们使穷人离开正道,世上的贫民尽都隐藏。

这些贫穷人,如同野驴出到旷野,殷勤寻梢食物。他们靠着野地给儿女糊口,

收割别人田间的禾稼,摘取恶人馀剩的葡萄。

终夜赤身无衣,天气寒冷毫无遮盖,

在山上被大雨淋湿,因没有避身之处就挨近磐石。

又有人从母怀中抢夺孤儿,强取穷人的衣服为当头。

使人赤身无衣,到处流行,且因饥饿扛抬禾捆,

在那些人的围墙内造油,榨酒,自己还口渴。

在多民的城内有人唉哼,受伤的人哀号。神却不理会(那恶人的愚妄。

又有人背弃光明,不认识光明的道,不住在光明的路上。

杀人的黎明起来,杀害困苦穷乏人,夜间又作盗贼。

奸夫等候黄昏,说,必无眼能见我,就把脸蒙蔽。

盗贼黑夜挖窟窿,白日躲藏,并不认识光明。

他们看早晨如幽暗,因为他们晓得幽暗的惊骇。

这些恶人犹如浮萍快快飘去。他们所得的分在世上被咒诅。他们不得再走葡萄园的路。

乾旱炎热消没雪水,阴间也如此消没犯罪之辈。

怀他的母(原文是胎)要忘记他。虫子要吃他,觉得甘甜。他不再被人记念。不义的人必如树折断。

他恶待(或作他吞灭)不怀孕不生养的妇人,不善待寡妇。

然而神用能力保全有势力的人,那性命难保的人仍然兴起。

神使他们安稳,他们就有所倚靠。神的眼目也看顾他们的道路。

他们被高举,不过片时就没有了。他们降为卑,被除灭,与众人一样,又如谷穗被割。

若不是这样,谁能证实我是说谎的,将我的言语驳为虚空呢。


\chapter{第25章}
书亚人比勒达回答说,

神有治理之权,有威严可畏。他在高处施行和平。

他的诸军,岂能数算。他的光亮一发,谁不蒙照呢。

这样在神面前,人怎能称义。妇人所生的怎能洁净。

在神眼前,月亮也无光亮,星宿也不清洁。

何况如虫的人,如蛆的世人呢。


\chapter{第26章}
约伯回答说,

无能的人,蒙你何等的帮助。膀臂无力的人,蒙你何等的拯救。

无智慧的人,蒙你何等的指教。你向他多显大知识。

你向谁发出言语来。谁的灵从你而出。

在大水,和水族以下的阴魂,战兢。

在神面前,阴间显露。灭亡也不得遮掩。

神将北极铺在空中,将大地悬在虚空。

将水包在密云中,云却不破裂。

遮蔽他的宝座,将云铺在其上。

在水面的周围划出界限,直到光明黑暗的交界。

天的柱子,因他的斥责,震动惊奇。

他以能力搅动(或作平静)大海,他藉知识打伤拉哈伯。

藉他的灵使天有妆饰,他的手刺杀快蛇。

看哪,这不过是神工作的些微。我们所听于他的,是何等细微的声音。他大能的雷声谁能明透呢。


\chapter{第27章}
约伯接着说,

神夺去我的理,全能者使我心中愁苦。我指着永生的神起誓。

我的生命尚在我里面,神所赐呼吸之气,仍在我的鼻孔内。

我的嘴决不说非义之言,我的舌也不说诡诈之语。

我断不以你们为是,我至死必不以自己为不正。

我持定我的义,必不放松。在世的日子,我心必不责备我。

愿我的仇敌如恶人一样。愿那起来攻击我的,如不义之人一般。

不敬虔的人虽然得利,神夺取其命的时候,还有什么指望呢。

患难临到他,神岂能听他的呼求。

他岂以全能者为乐,随时求告神呢。

神的作为,我要指教你们。全能者所行的,我也不隐瞒。

你们自己也都见过,为何全然变为虚妄呢。

神为恶人所定的分,强暴人从全能者所得的报(报原文作产业)乃是这样。

倘或他的儿女增多,还是被刀所杀。他的子孙必不得饱食。

他所遗留的人必死而埋葬,他的寡妇也不哀哭。

他虽积蓄银子如尘沙,预备衣服如泥土。

他只管预备,义人却要穿上。他的银子,无辜的人要分取。

他建造房屋如虫做窝,又如守望者所搭的棚。

他虽富足躺卧,却不得收殓,转眼之间就不在了。

惊恐如波涛将他追上。暴风在夜间将他刮去。

东风把他飘去,又刮他离开本处。

神要向他射箭,并不留情。他恨不得逃脱神的手。

人要向他拍掌,并要发叱声,使他离开本处。


\chapter{第28章}
银子有矿,炼金有方。

铁从地里挖出,铜从石中熔化。

人为黑暗定界限,查究幽暗阴翳的石头,直到极处。

在无人居住之处刨开矿穴,过路的人也想不到他们。又与人远离,悬在空中摇来摇去。

至于地,能出粮食,地内好像被火翻起来。

地中的石头有蓝宝石,并有金沙。

矿中的路鸷鸟不得知道,鹰眼也未见过。

狂傲的野兽未曾行过。猛烈的狮子也未曾经过。

人伸手凿开坚石,倾倒山根。

在磐石中凿出水道,亲眼看见各样宝物。

他封闭水不得滴流,使隐藏的物显露出来。

然而,智慧有何处可寻。聪明之处在那里呢。

智慧的价值无人能知,在活人之地也无处可寻。

深渊说,不在我内。沧海说,不在我中。

智慧非用黄金可得,也不能平白银为它的价值。

俄斐金,和贵重的红玛瑙,并蓝宝石,不足与较量。

黄金,和玻璃,不足与比较。精金的器皿,不足与兑换。

珊瑚,水晶都不足论。智慧的价值胜过珍珠(或作红宝石)。

古实的红璧玺,不足与比较。精金,也不足与较量。

智慧从何处来呢。聪明之处在那里呢。

是向一切有生命的眼目隐藏,向空中的飞鸟掩蔽。

灭没和死亡说,我们风闻其名。

神明白智慧的道路,晓得智慧的所在。

因他鉴察直到地极,遍观普天之下。

要为风定轻重,又度量诸水。

他为雨露定命令,为雷电定道路。

那时他看见智慧,而且述说。他坚定,并且查究。

他对人说,敬畏主就是智慧。远离恶便是聪明。


\chapter{第29章}
约伯又接着说,

惟愿我的景况如从前的月份,如神保守我的日子。

那时他的灯照在我头上。我藉他的光行过黑暗。

我愿如壮年的时候,那时我在帐棚中。神待我有密友之情。

全能者仍与我同在。我的儿女都环绕我。

奶多可洗我的脚。磐石为我出油成河。

我出到城门,在街上设立座位。

少年人见我而回避,老年人也起身站立。

王子都停止说话,用手糊口。

首领静默无声,舌头贴住上膛。

耳朵听我的,就称我有福。眼睛看我的,便称赞我。

因我拯救哀求的困苦人,和无人帮助的孤儿。

将要灭亡的为我祝福。我也使寡妇心中欢乐。

我以公义为衣服,以公平为外袍和冠冕。

我为瞎子的眼,瘸子的脚。

我为穷乏人的父,素不认识的人,我查明他的案件。

我打破不义之人的牙床,从他牙齿中夺了所抢的。

我便说,我必死在家中(原文作窝中),必增添我的日子,多如尘沙。

我的根长到水边,露水终夜沾在我的枝上。

我的荣耀在身上增新,我的弓在手中日强。

人听见我而仰望,静默等候我的指教。

我说话之后,他们就不再说。我的言语像雨露滴在他们身上。

他们仰望我如仰望雨,又张开口如切慕春雨。

他们不敢自信,我就向他们含笑。他们不使我脸上的光改变。

我为他们选择道路,又坐首位。我如君王在军队中居住,又如吊丧的安慰伤心的人。


\chapter{第30章}
但如今,比我年少的人戏笑我。其人之父我曾藐视,不肯安在看守我羊群的狗中。

他们壮年的气力既已衰败,其手之力与我何益呢。

他们因穷乏饥饿,身体枯瘦,在荒废凄凉的幽暗中龈乾燥之地。

在草丛之中采咸草,罗腾(罗腾小树名松类)的根为他们的食物。

他们从人中被赶出,人追喊他们如贼一般。

以致他们住在荒谷之间,在地洞和岩穴中。

在草丛中叫唤,在荆棘下聚集。

这都是愚顽下贱人的儿女,他们被鞭打,赶出境外。

现在这些人以我为歌曲,以我为笑谈。

他们厌恶我,躲在旁边站着,不住地吐唾沫在我脸上。

松开他们的绳索苦待我,在我面前脱去辔头。

这等下流人在我右边起来,推开我的脚,筑成战路来攻击我。

这些无人帮助的,毁坏我的道,加增我的灾。

他们来如同闯进大破口,在毁坏之间滚在我身上。

惊恐临到我,驱逐我的尊荣如风,我的福禄如云过去。

现在我心极其悲伤。困苦的日子将我抓住。

夜间,我里面的骨头刺我,疼痛不止,好像龈我。

因神的大力,我的外衣污秽不堪,又如里衣的领子将我缠住。

神把我扔在淤泥中,我就像尘土和炉灰一般。

主阿,我呼求你,你不应允我。我站起来,你就定睛看我。

你向我变心,待我残忍,又用大能追逼我。

把我提在风中,使我驾风而行,又使我消灭在烈风中。

我知道要使我临到死地,到那为众生所定的阴宅。

然而,人仆倒岂不伸手。遇灾难岂不求救呢。

人遭难,我岂不为他哭泣呢。人穷乏,我岂不为他忧愁呢。

我仰望得好处,灾祸就到了。我等待光明,黑暗便来了。

我心里烦扰不安,困苦的日子临到我身。

我没有日光就哀哭行去(或作我面发黑并非因日晒)。我在会中站着求救。

我与野狗为弟兄,与鸵鸟为同伴。

我的皮肤黑而脱落,我的骨头因热烧焦。

所以,我的琴音变为悲音,我的箫声变为哭声。


\chapter{第31章}
我与眼睛立约,怎能恋恋瞻望处女呢。

从至上的神所得之分,从至高全能者所得之业,是什么呢。

岂不是祸患临到不义的,灾害临到作孽的呢。

神岂不是察看我的道路,数点我的脚步呢。

我若与虚谎同行,脚若追随诡诈。

我若被公道的天平称度,使神可以知道我的纯正。

我的脚步若偏离正路,我的心若随着我的眼目,若有玷污粘在我手上。

就愿我所种的有别人吃,我田所产的被拔出来。

我若受迷惑,向妇人起淫念,在邻舍的门外蹲伏。

就愿我的妻子给别人推磨,别人也与她同室。

因为这是大罪,是审判官当罚的罪孽。

这本是火焚烧,直到毁灭,必拔除我所有的家产。

我的仆婢与我争辩的时候,我若藐视不听他们的情节。

神兴起,我怎样行呢。他察问,我怎样回答呢。

造我在腹中的,不也是造他吗。将他与我抟在腹中的,岂不是一位吗。

我若不容贫寒人得其所愿,或叫寡妇眼中失望,

或独自吃我一点食物,孤儿没有与我同吃。

(从幼年时孤儿与我同长,好像父子一样。我从出母腹就扶助寡妇)。(扶助原文作引领)

我若见人因无衣死亡,或见穷乏人身无遮盖。

我若不使他因我羊的毛得暖,为我祝福。

我若在城门口见有帮助我的,举手攻击孤儿。

情愿我的肩头从缺盆骨脱落,我的膀臂从羊矢骨折断。

因神降的灾祸使我恐惧。因他的威严,我不能妄为。

我若以黄金为指望,对精金说,你是我的倚靠。

我若因财物丰裕,因我手多得资财而欢喜。

我若见太阳发光,明月行在空中,

心就暗暗被引诱,口便亲手。

这也是审判官当罚的罪孽,又是我背弃在上的神。

我若见恨我的遭报就欢喜,见他遭灾便高兴。

(我没有容口犯罪,咒诅他的生命)

若我帐棚的人未尝说,谁不以主人的食物吃饱呢。

(从来我没有容客旅在街上住宿,却开门迎接行路的人)

我若像亚当(或作别人)遮掩我的过犯,将罪孽藏在怀中。

因惧怕大众,又因宗族藐视我使我惊恐,以致闭口无言,杜门不出。

惟愿有一位肯听我。(看哪,在这里有我所划的押,愿全能者回答我)

愿那敌我者,所写的状词在我这里,我必带在肩上,又绑在头上为冠冕。

我必向他述说我脚步的数目,必如君王进到他面前。

我若夺取田地,这地向我喊冤,犁沟一同哭泣。

我若吃地的出产不给价值,或叫原主丧命。

愿这地长蒺藜代替麦子,长恶草代替大麦。约伯的话说完了。


\chapter{第32章}
于是这三个人,因约伯自以为义就,不再回答他。

那时有布西人,兰族巴拉迦的儿子,以利户向约伯发怒。因约伯自以为义,不以神为义。

他又向约伯的三个朋友发怒。因为他们想不出回答的话来,仍以约伯为有罪。

以利户要与约伯说话,就等候他们,因为他们比自己年老。

以利户见这三个人口中无话回答,就怒气发作。

布西人,巴拉迦的儿子,以利户回答说,我年轻,你们老迈。因此我退让,不敢向你们陈说我的意见。

我说,年老的当先说话。寿高的当以智慧教训人。

但在人里面有灵,全能者的气使趟有聪明。

尊贵的不都有智慧。寿高的不都能明白公平。

因此我说,你们要听我言,我也要陈说我的意见。

你们查究所要说的话。那时我等候你们的话,侧耳听你们的辩论。

留心听你们。谁知你们中间无一人折服约伯,驳倒他的话。

你们切不可说,我们寻得智慧。神能胜他,人却不能。

约伯没有向我争辩。我也不用你们的话回答他。

他们惊奇不再回答,一言不发。

我岂因他们不说话,站住不再回答,仍旧等候呢。

我也要回答我的一分话,陈说我的意见。

因为我的言语满怀。我里面的灵激动我。

我的胸怀如盛酒之囊,没有出气之缝,又如新皮袋快要破裂。

我要说话,使我舒畅。我要开口回答。

我必不看人的情面,也不奉承人。

我不晓得奉承。若奉承,造我的主必快快除灭我。


\chapter{第33章}
约伯阿,请听我的话,留心听我一切的言语。

我现在开口,用舌发言。

我的言语要发明心中所存的正直。我所知道的,我嘴唇要诚实地说出。

神的灵造我,全能者的气使我得生。

你若回答我,就站起来,在我面前陈明。

我在神面前与你一样,也是用土造成。

我不用威严惊吓你,也不用势力重压你。

你所说的,我听见了,也听见你的言语,说,

我是清洁无过的,我是无辜的。在我里面也没有罪孽。

神找机会攻击我,以我为仇敌,

把我的脚上了木狗,窥察我一切的道路。

我要回答你说,你这话无理,因神比世人更大。

你为何与他争论呢。因他的事都不对人解说。

神说一次,两次,世人却不理会。

人躺在床上沉睡的时候,神就用梦,和夜间的异象,

开通他们的耳朵,将当受的教训印在他们心上,

好叫人不从自己的谋算,不行骄傲的事。(原文作将骄傲向人隐藏)

拦阻人不陷于坑里,不死在刀下。

人在床上被惩治,骨头中不住地疼痛。

以致他的口厌弃食物,心厌恶美味。

他的肉消瘦,不得再见。先前不见的骨头都凸出来。

他的灵魂临近深坑。他的生命近于灭命的。

一千天使中,若有一个作传话的与神同在,指示人所当行的事。

神就给他开恩,说,救赎他免得下坑。我已经得了赎价。

他的肉要比孩童的肉更嫩。他就返老还童。

他祷告神,神就喜悦他,使他欢呼朝见神的面。神又看他为义。

他在人前歌唱说,我犯了罪,颠倒是非,这竟与我无益。

神救赎我的灵魂免入深坑。我的生命也必见光。

神两次,三次,向人行这一切的事。

为要从深坑救回人的灵魂,使他被光照耀与活人一样。

约伯阿,你当侧耳听我的话,不要作声,等我讲说。

你若有话说,就可以回答我。你只管说,因我愿以你为是。

若不然,你就听我说。你不要作声,我便将智慧教训你。


\chapter{第34章}
以利户又说,

你们智慧人,要听我的话。有知识的人,要留心听我说。

因为耳朵试验话语,好像上膛尝食物。

我们当选择何为是,彼此知道何为善。

约伯曾说,我是公义,神夺去我的理。

我虽有理,还算为说谎言的。我虽无过,受的伤还不能医治。

谁像约伯,喝讥诮如同喝水呢。

他与作孽的结伴,和恶人同行。

他说,人以神为乐,总是无益。

所以你们明理的人,要听我的话。神断旁不至行恶,全能者断不至作孽。

他必按人所作的报应人,使各人照所行的得报。

神必不作恶,全能者也不偏离公平。

谁派他治理地,安定全世界呢。

他若专心为己,将灵和气收归自己。

凡有血气的就必一同死亡,世人必仍归尘土。

你若明理,就当听我的话,留心听我言语的声音。

难道恨恶公平的可以掌权吗。那有公义的,有大能的,岂可定他有罪吗。

他对君王说,你是鄙陋的。对贵臣说,你是邪恶的。

他待王子不徇情面,也不看重富足的过于贫穷的,因为都是他手所造。

在转眼之间,半夜之中,他们就死亡。百姓被震动而去世。有权力的被夺去非借人手。

神注目观看人的道路,看明人的脚步。

没有黑暗,阴翳能给作孽的藏身。

神审判人,不必使人到他面前再三鉴察。

他用难测之法,打破有能力的人,设立别人代替他们。

他原知道他们的行为,使他们在夜间倾倒灭亡。

他在众人眼前击打他们,如同击打恶人一样。

因为他们偏行不跟从他,也不留心他的道,

甚至使贫穷人的哀声,达到他那里。他也听了困苦人的哀声。

他使人安静,谁能扰乱(或作定罪)呢,他掩面谁能见他呢。无论待一国,或一人都是如此。

使不虔敬的人不得作王,免得有人牢宠百姓。

有谁对神说,我受了责罚,不再犯罪。

我所看不明的,求你指教我。我若作了孽,必不再作。

他施行报应,岂要随你的心愿,叫你推辞不受吗。选定的是你,不是我。你所知道的只管说吧。

明理的人,和听我话的智慧人,必对我说。

约伯说话没有知识,言语中毫无智慧。

愿约伯被试验到底,因他回答像恶人一样。

他在罪上又加悖逆。在我们中间拍手,用许多言语轻慢神。


\chapter{第35章}
以利户又说,

你以为有理,或以为你的公义胜于神的公义,

才说这与我有什么益处。我不犯罪比犯罪有什么好处呢。

我要回答你,和在你这里的朋友。

你要向天观看,瞻望那高于你的穹苍。

你若犯罪,能使神受何害呢。你的过犯加增,能使神受何损呢。

你若是公义,还能加增他什么呢。他从你手里还接受什么呢。

你的过恶,或能害你这类的人。你的公义,或能叫世人得益处。

人因多受欺压就哀求,因受能者的辖制(辖制原文作膀臂)便求救。

却无人说,造我的神在那里。他使人夜间歌唱。

教训我们胜于地上的走兽,使我们有聪明胜于空中的飞鸟。

他们在那里,因恶人的骄傲呼求,却无人答应。

虚妄的呼求,神必不垂听。全能者也必不眷顾。

何况你说,你不得见他。你的案件在他面前,你等候他吧。

但如今因他未曾发怒降罚,也不甚理会狂傲,

所以约伯开口说虚妄的话,多发无知识的言语。


\chapter{第36章}
以利户又接着说,

你再容我片时,我就指示你,因我还有话为神说。

我要将所知道的从远处引来,将公义归给造我的主。

我的言语真不虚谎。有知识全备的与你同在。

神有大能,并不藐视人。他的智慧甚广。

他不保护恶人的性命,却为困苦人伸冤。

他时常看顾义人,使他们和君王同坐宝座,永远要被高举。

他们若被锁链捆住,被苦难的绳索缠住,

他就把他们的作为,和过犯指示他们,叫他们知道有骄傲的行动。

他也开通他们的耳朵,得受教训,吩咐他们离开罪孽转回。

他们若听从事奉他,就必度日亨通,历年福乐。

若不听从,就要被刀杀灭,无知无识而死。

那心中不敬虔的人积蓄怒气。神捆绑他们,他们竟不求救。

必在青年时死亡,与污秽人一样丧命。

神藉着困苦救拔困苦人,趁他们受欺压,开通他们的耳朵。

神也必引你出离患难,进入宽阔不狭窄之地。摆在你席上的必满有肥甘。

但你满口有恶人批评的言语。判断和刑罚抓住你。

不可容忿怒触动你,使你不服责罚。也不可因赎价大就偏行。

你的呼求(或作资财),或是你一切的势力,果有灵验,叫你不受患难吗。

不要切慕黑夜,就是众民在本处被除灭的时候。

你要谨慎,不可重看罪孽,因你选择罪孽,过于选择苦难。

神行事有高大的能力。教训人的有谁像他呢。

谁派定他的道路。谁能说,你所行的不义。

你不可忘记称赞他所行的为大,就是人所歌颂的。

他所行的,万人都看见,世人也从远处观看。

神为大,我们不能全知,他的年数不能测度。

他吸取水点,这水点从云雾中就变成雨。

云彩将雨落下,沛然降与世人。

谁能明白云彩如何铺张,和神行宫的雷声呢。

他将亮光普照在自己的四围。他又遮覆海底。

他用这些审判众民,且赐丰富的粮食。

他以电光遮手,命闪电击中敌人(或作中了靶子)。

所发的雷声显明他的作为,又向牲畜指明要起暴风。


\chapter{第37章}
因此我心战兢,从原处移动。

听阿,神轰轰的声音,是他口中所发的响声。

他发响声震遍天下,发电光闪到地极。

随后人听见有雷声轰轰,大发威严,雷电接连不断。

神发出奇妙的雷声,他行大事,我们不能测透。

他对雪说,要降在地上,对大雨和暴雨也是这样说。

他封住各人的手,叫所造的万人,都晓得他的作为。

百兽进入穴中,卧在洞内。

暴风出于南宫。寒冷出于北方。

神嘘气成冰。宽阔之水也都凝结。

他使密云盛满水气,布散电光之云。

这云是藉他的指引游行旋转,得以在全地面上,行他一切所吩咐的,

或为责罚,或为润地,或为施行慈爱。

约伯阿,你要留心听,要站立思想神奇妙的作为。

神如何吩咐这些,如何使云中的电光照耀,你知道吗。

云彩如何浮于空中,那知识全备者奇妙的作为,你知道吗。

南风使地寂静,你的衣服就如火热,你知道吗。

你岂能与神同铺穹苍吗。这穹苍坚硬,如同铸成的镜子。

我们愚昧不能陈说。请你指教我们该对他说什么话。

人岂可说,我愿与他说话。岂有人自愿灭亡吗。

现在有云遮蔽,人不得见穹苍的光亮。但风吹过,天又发晴。

金光出于北方,在神那里有可怕的威严。

论到全能者,我们不能测度。他大有能力,有公平和大义,必不苦待人。

所以,人敬畏他。凡自以为心中有智慧的人,他都不顾念。


\chapter{第38章}
那时,耶和华从旋风中回答约伯说,

谁用无知的言语,使我的旨意暗昧不明。

你要如勇士束腰。我问你,你可以指示我。

我立大地根基的时候,你在那里呢。你若有聪明,只管说吧。

你若晓得就说,是谁定地的尺度。是谁把准绳拉在其上。

地的根基安置在何处。地的角石是谁安放的。

那时晨星一同歌唱,神的众子也都欢呼。

海水冲出,如出胎胞,那时谁将它关闭呢。

是我用云彩当海的衣服,用幽暗当包裹它的布,

为它定界限,又安门和闩,

说,你只可到这里,不可越过。你狂傲的浪要到此止住。

你自生以来,曾命定晨光,使清晨的日光知道本位。

叫这光普照地的四极,将恶人从其中驱逐出来吗。

因这光,地面改变如泥上印印,万物出现如衣服一样。

亮光不照恶人,强横的膀臂也必折断。

你曾进到海源,或在深渊的隐密处行走吗。

死亡的门曾向你显露吗。死荫的门你曾见过吗。

地的广大你能明透吗。你若全知道,只管说吧。

光明的居所从何而至。黑暗的本位在于何处。

你能带到本境,能看明其室之路吗。

你总知道,因为你早已生在世上,你日子的数目也多。

你曾进入雪库。或见过雹仓吗。

这雪雹乃是我为降灾,并打仗和争战的日子所预备的。

光亮从何路分开。东风从何路分散遍地。

谁为雨水分道。谁为雷电开路。

使雨降在无人之地,无人居住的旷野。

使荒废凄凉之地得以丰足,青草得以发生。

雨有父吗。露水珠是谁生的呢。

冰出于谁的胎。天上的霜是谁生的呢。

诸水坚硬(或作隐藏)如石头,深渊之面凝结成冰。

你能系住昴星的结吗。能解开叁星的带吗。

你能按时领出十二宫吗。能引导北斗和随它的众星(星原文作子)吗。

你知道天的定例吗。能使地归在天的权下吗。

你能向云彩扬起声来,使倾盆的雨遮盖你吗。

你能发出闪电,叫它行去,使它对你说,我们在这里。

谁将智慧放在怀中。谁将聪明赐于心内。

谁能用智慧数算云彩呢。尘土聚集成团,土块紧紧结连。

那时,谁能倾倒天上的瓶呢。

母狮子在洞中蹲伏,少壮狮子在隐密处埋伏。

你能为它们抓取食物,使它们饱足吗。

乌鸦之雏因无食物飞来飞去,哀告神。那时,谁为它预备食物呢。


\chapter{第39章}
山岩间的野山羊几时生产,你知道吗。母鹿下犊之期,你能察定吗。

它们怀胎的月数,你能数算吗。它们几时生产,你能晓得吗。

它们屈身,将子生下,就除掉疼痛。

这子渐渐肥壮,在荒野长大,去而不回。

谁放野驴出去自由。谁解开快驴的绳索。

我使旷野作它的住处,使咸地当它的居所。

它嗤笑城内的喧囔,不听赶牲口的喝声。

遍山是它的草场。它寻梢各样青绿之物。

野牛岂肯服事你。岂肯住在你的槽旁。

你岂能用套绳将野牛笼在犁沟之间。它岂肯随你耙山谷之地。

岂可因它的力大就倚靠它。岂可把你的工交给它做吗。

岂可信靠它把你的粮食运到家,又收聚你禾场上的谷吗。

鸵鸟的翅膀欢然扇展,岂是显慈爱的翎毛和羽毛吗。

因它把蛋留在地上,在尘土中使得温暖。

却想不到被脚踹碎,或被野兽践踏。

它忍心待雏,似乎不是自己的。虽然徒受劳苦,也不为雏惧怕。

因为神使它没有智慧,也未将悟性赐给它。

它几时挺身展开翅膀,就嗤笑马和骑马的人。

马的大力是你所赐的吗。它颈项上??挲的鬃是你给它披上的吗。

是你叫它跳跃像蝗虫吗。它喷气之威使人惊惶。

它在谷中刨地,自喜其力。它出去迎接佩带兵器的人。

它嗤笑可怕的事并不惊惶,也不因刀剑退回。

箭袋和发亮的枪,并短枪在它身上铮铮有声。

它发猛烈的怒气将地吞下。一听角声就不耐站立。

角每发声,它说呵哈。它从远处闻着战气,又听见军长大发雷声,和兵丁呐喊。

鹰雀飞翔,展开翅膀一直向南,岂是藉你的智慧吗。

大鹰上腾在高处搭窝,岂是听你的吩咐吗。

它住在山岩,以山峰和坚固之所为家,

从那里窥看食物,眼睛远远观望。

它的雏也咂血。被杀的人在那里。它也在那里。


\chapter{第40章}
耶和华又对约伯说,

强辩的岂可与全能者争论吗。与神辩驳的可以回答这些吧。

于是,约伯回答耶和华说,

我是卑贱的。我用什么回答你呢。只好用手捂口。

我说了一次,再不回答。说了两次,就不再说。

于是,耶和华从旋风中回答约伯说,

你要如勇士束腰。我问你,你可以指示我。

你岂可废弃我所拟定的。岂可定我有罪,好显自己为义吗。

你有神那样的膀臂吗。你能像他发雷声吗。

你要以荣耀庄严为妆饰,以尊荣威严为衣服。

要发出你满溢的怒气,见一切骄傲的人,使他降卑。

见一切骄傲的人,将他制伏,把恶人践踏在本处。

将他们一同隐藏在尘土中,把他们的脸蒙蔽在隐密处。

我就认你右手能以救自己。

你且观看河马。我造你也造它。它吃草与牛一样。

它的气力在腰间,能力在肚腹的筋上。

它摇动尾巴如香柏树。它大腿的筋互相联络。

它的骨头好像铜管。它的肢体彷佛铁棍。

它在神所造的物中为首。创造它的给它刀剑。

诸山给它出食物,也是百兽游玩之处。

它伏在莲叶之下,卧在芦苇隐密处和水洼子里。

莲叶的阴凉遮蔽它。溪旁的柳树环绕它。

河水泛滥,它不发战。就是约旦河的水涨到它口边,也是安然。

在它防备的时候,谁能捉拿它。谁能牢笼它穿它的鼻子呢。


\chapter{第41章}
你能用鱼钩钓上鳄鱼吗。能用绳子压下它的舌头吗。

你能用绳索穿它的鼻子吗。能用钩穿它的腮骨吗。

它岂向你连连恳求,说柔和的话吗。

岂肯与你立约,使你拿它永远作奴仆吗。

你岂可拿它当雀鸟玩耍吗。岂可为你的幼女将它拴住吗。

搭夥的渔夫岂可拿它当货物吗。能把它分给商人吗。

你能用倒钩枪扎满它的皮,能用鱼叉叉满它的头吗。

你按手在它身上,想与它争战,就不再这样行吧。

人指望捉拿它是徒然的。一见它,岂不丧胆吗。

没有那吗凶猛的人敢惹它。这样,谁能在我面前删立得住呢。

谁先给我什么,使我偿还呢。天下万物都是我的。

论到鳄鱼的肢体和其大力,并美好的骨骼,我不能缄默不言。

谁能剥它的外衣。谁能进它上下牙骨之间呢。

谁能开它的腮颊。它牙齿四围是可畏的。

它以坚固的鳞甲为可夸,紧紧合闭,封得严密。

这鳞甲一一相连,甚至气不得透入其间,

都是互相联络,胶结,不能分离。

它打喷嚏就发出光来。它眼睛好像早晨的光线(原文作眼皮)。

从它口中发出烧着的火把,与飞迸的火星。

从它鼻孔冒出烟来,如烧开的锅和点着的芦苇。

它的气点着煤炭,有火焰从它口中发出。

它颈项中存着劲力。在它面前的都恐吓蹦跳。

它的肉块互相联络,紧贴其身,不能摇动。

它的心结实如石头,如下磨石那样结实。

它一起来,勇士都惊恐,心里慌乱,便都昏迷。

人若用刀,用枪,用标枪,用尖枪扎它,都是无用。

它以铁为乾草,以铜为烂木。

箭不能恐吓它使它逃避。弹石在它看为碎秸。

棍棒算为禾秸。它嗤笑短枪飕的响声。

它肚腹下如尖瓦片,它如钉耙经过淤泥。

它使深渊开滚如锅,使洋海如锅中的膏油。

它行的路随后发光,令人想深渊如同白发。

在地上没有像它造的那样,无所惧怕。

凡高大的,它无不藐视。它在骄傲的水族上作王。


\chapter{第42章}
约伯回答耶和华说,

我知道,你万事都能做。你的旨意不能拦阻。

谁用无知的言语,使你的旨意隐藏呢。我所说的,是我不明白的。这些事太奇妙,是我不知道的。

求你听我,我要说话。我问你,求你指示我。

我从前风闻有你,现在亲眼看见你。

因此我厌恶自己,(自己或作我的言语)在尘土和炉灰中懊悔。

耶和华对约伯说话以后,就对提幔人以利法说,我的怒气向你和你两个朋友发作,因为你们议论我,不如我的仆人约伯说的是。

现在你们要取七只公牛,七只公羊,到我仆人约伯那里去,为自己献上燔祭,我的仆人约伯就为你们祈祷。我因悦纳他,就不按你们的愚妄办你们。你们议论我,不如我的仆人约伯说的是。

于是提幔人以利法,书亚人比勒达,拿玛人琐法照着耶和华所吩咐的去行。耶和华就悦纳约伯。

约伯为他的朋友祈祷。耶和华就使约伯从苦境(原文作掳掠)转回,并且耶和华赐给他的,比他从前所有的加倍。

约伯的弟兄,姊妹,和以先所认识的人都来见他,在他家里一同吃饭。又论到耶和华所降与他的一切灾祸,都为他悲伤安慰他。每人也送他一块银子和一个金环。

这样,耶和华后来赐福给约伯比先前更多。他有一万四千羊,六千骆驼,一千对牛,一千母驴。

他也有七个儿子,三个女儿。

他给长女起名叫耶米玛,次女叫基洗亚,三女叫基连哈朴。

在那全地的妇女中,找不着像约伯的女儿那样美貌。她们的父亲使她们在弟兄中得产业。

此后,约伯又活了一百四十年,得见他的儿孙,直到四代。

这样,约伯年纪老迈,日子满足而死。


\end{document}