\documentclass[12pt,oneside]{book}

\usepackage{mybook}
\usepackage{mybookcover}

\hypersetup{
    pdftitle={新约圣经四福音},
    pdfauthor={和合本},
    pdfsubject={文学},
}

\title{新约圣经四福音}
\author{和合本}

\begin{document}
\bookcover{book_cover.png}
\flypage{感谢上帝}

\frontmatter


\addchtoc{目录}
\setcounter{tocdepth}{2}    
\tableofcontents

\mainmatter


\part{约翰福音}
\chapter{第1章}
太初有道,道与神同在,道就是神。

这道太初与神同在。

万物是藉着他造的。凡被造的,没有一样不是藉着他造的。

生命在他里头。这生命就是人的光。

光照在黑暗里,黑暗却不接受光。

有一个人,是从神那里差来的,名叫约翰。

这人来,为要作见证,就是为光作见证,叫众人因他可以信。

他不是那光,乃是要为光作见证。

那光是真光,照亮一切生在世上的人。

他在世界,世界也是藉着他造的,世界却不认识他。

他到自己的地方来,自己的人倒不接待他。

凡接待他的,就是信他名的人,他就赐他们权柄,作神的儿女。

这等人不是从血气生的,不是从情欲生的,也不是从人意生的,乃是从神生的。

道成了肉身住在我们中间,充充满满的有恩典有真理。我们也见过他的荣光,正是父独生子的荣光。

约翰为他作见证,喊着说,这就是我曾说,那在我以后来的,反成了在我以前的。因他本来在我以前。

从他丰满的恩典里我们都领受了,而且恩上加恩。

律法本是藉着摩西传的,恩典和真理,都是由耶稣基督来的。

从来没有人看见神。只有在父怀里的独生子将他表明出来。

约翰所作的见证,记在下面。犹太人从耶路撒冷差祭司和利未人到约翰那里,问他说,你是谁。

他就明说,并不隐瞒。明说,我不是基督。

他们又问他说,这样你是谁呢,是以利亚吗。他说,我不是。是那先知吗,他回答说,不是。

于是他们说,你到底是谁,叫我们好回覆差我们来的人。你自己说,你是谁。

他说,我就是那在旷野有人声喊着说,修直主的道路,正如以赛亚所说的。

那些人是法利赛人差来的。(或作那差来的是法利赛人)

他们就问他说,你既不是基督,不是以利亚,也不是那先知,为什么施洗呢。

约翰回答说,我是用水施洗,但有一位站在你们中间,是你们不认识的,

就是那在我以后来的,我给他解鞋带,也不配。

这是在约旦河外,伯大尼,(有古卷作伯大巴喇)约翰施洗的地方作的见证。

次日,约翰看见耶稣来到他那里,就说,看哪,神的羔羊,除去(或作背负)世人罪孽的。

这就是我曾说,有一位在我以后来,反成了在我以前的。因他本来在我以前。

我先前不认识他,如今我用水施洗,为要叫他显明给以色列人。

约翰又作见证说,我曾看见圣灵,彷佛鸽子从天降下,住在他的身上。

我先前不认识他。只是那差我来用水施洗的,对我说,你看见圣灵降下来,住在谁的身上,谁就是用圣灵施洗的。

我看见了,就证明这是神的儿子。

再次日,约翰同两个门徒站在那里。

他见约稣行走,就说,看哪,这是神的羔羊。

两个门徒听见他的话,就跟从了耶稣。

耶稣转过身来,看见见他们跟着,就问他们说,你们要什么。他们说,拉比,在那里住。(拉比翻出来,就是夫子

耶稣说,你们来看。他们就去看他在那里住,这一天便与他同住,那时约有申正了。

听见约翰的话,跟从耶稣的那两个人,一个是西门彼得的兄弟安得烈。

他先找着自己的哥哥西门,对他说,我们遇见弥赛亚了,(弥赛亚翻出来,就是基督)

于是领他去见耶稣。耶稣看见他说,你是约翰的儿子西门,(约翰马太十六十七节称约拿)你要称为矶法。(矶法翻出来,就是彼得)

又次日,耶稣想要往加利利去,遇见腓力,就对他说,来跟从我吧。

这腓力是伯赛大人,和安得烈同城。

腓力找着拿但业,对他说,摩西在律法上所写的,和众先知所记的那一位,我们遇见了,就是约瑟的儿子拿撒勒人耶稣。

拿但业对他说,拿撒勒还能出什么好的吗。腓力说,你来看。

耶稣看见拿但业来,就指着他说,看哪,这是个真以色列人,他心里是没有诡诈的。

拿但业对耶稣说,你从那里知道我呢。耶稣回答说,腓力还没有招呼你,你在无花果树底下,我就看见你了。

拿但业说,拉比,你是神的儿子,你是以色列的王。

耶稣对他说,因为我说在无花果树底下看见你,你就信吗。你将要看见比这更大的事。

又说,我实实在在的告诉你们,你们将要看见天开了,神的使者上去下来在人子身上。

\chapter{第2章}
第三日,在加利利的迦拿有娶亲的筵席。耶稣的母亲在那里。

耶稣和他的门徒也被请去赴席。

酒用尽了,耶稣的母亲对他说,他们没有酒了。

耶稣说,母亲,(原文作妇人)我与你有什么相干。我的时候还没有到。

他母亲对用人说,他告诉你们什么,你们就作什么。

照犹太人洁净的规矩,有六口缸摆在那里,每口可以盛两三桶水。

耶稣对用人说,把缸倒满了水。他们就倒满了,直到缸口。

耶稣又说,现在可以舀出来,送给管筵席的。他们就送了去。

管筵席尝了那水变的酒,并不知道是那里来的,只有舀水的用人知道。管筵席的便叫新郎来。

对他说,人都是先摆上好酒。等客喝足了,才摆上次的。你倒把好酒留到如今。

这是耶稣所行的头一件神迹,是在加利利的迦拿行的,显出他的荣耀来。他的门徒就信他了。

这事以后,耶稣与他的母亲弟兄和门徒,都下迦百农去。在那里住了不多几日。

犹太人的逾越节近了,耶稣就上耶路撒冷去。

看见殿里有卖牛羊鸽子的,并有兑换银钱的人,坐在那里。

耶稣就拿绳子作成鞭子,把牛羊都赶出殿去。倒出兑换银钱之人的银钱,推翻他们的桌子。

又对卖鸽子的说,把这些东西拿去。不要将我父的殿,当作买卖的地方。

他的门徒就想起经上记着说,我为你的殿,心里焦急,如同火烧。

因此,犹太人问他说,你既作这些事,还显什么神迹给我们看呢。

耶稣回答说,你们拆毁这殿,我三日内要再建立起来。

犹太人便说,这殿是四十六年才造成的,你三日内就再建立起来吗。

但耶稣这话,是以他的身体为殿。

所以到他从死里复活以后,门徒就想起他说过这话,便信了圣经和耶稣所说的。

当耶稣在耶路撒冷过逾越节的时候,有许多人看见他所行的神迹,就信了他的名。

耶稣却不将自己交托他们,因为他知道万人。

也用不着谁见证人怎样。因他知道人心里所存的。

\chapter{第3章}
有一个法利赛人,名叫尼哥底母,是犹太人的官。

这人夜里来见耶稣,说,拉比,我们知道你是由神那里来作师傅的。因为你所行的神迹,若没有神同在,无人能行。

耶稣回答说,我实实在在的告诉你,人若不重生,就不能见神的国。

尼哥底母说,人已经老了,如何能重生呢。岂能再进母腹生出来吗。

耶稣说,我实实在在的告诉你,人若不是从水和圣灵生的,就不能进神的国。

从肉身生的,就是肉身。从灵生的,就是灵。

我说,你们必须重生,你不要以为希奇。

风随着意思吹,你听见风的响声,却不晓得从那里来,往那里去。凡从圣灵生的,也是如此。

尼哥底母问他说,怎能有这事呢。

耶稣回答说,你是以色列人的先生,还不明白这事吗。

我实实在在的告诉你,我们所说的,是我们知道的,我们所见证的,是我们见过的。你们却不领受我们的见证。

我对你们说地上的事,你门尚且不信,若说天上的事,如何能信呢。

除了从天降下仍旧在天的人子,没有人升过天。

摩西在旷野怎样举蛇,人子也必照样被举起来。

叫一切信他的都得永生。(或作叫一切信的人在他里面得永生)

神爱世人,甚至将他的独生子赐给他们,叫一切信他的,不至灭亡,反得永生。

因为神差他的儿子降世,不是要定世人的罪,(或作审判世人下同)乃是要叫世人因他得救。

信他的人,不被定罪。不信的人,罪已经定了,因为他不信神独生子的名。

光来到世间,世人因自己的行为是恶的,不爱光倒爱黑暗,定他们的罪就是在此。

凡作恶的便恨光,并不来就光,恐怕他的行为受责备。

但行真理的必来就光,要显明他所行的是靠神而行。

这事以后,耶稣和门徒到了犹太地,在那里居住施洗。

约翰在靠近撒冷的哀嫩也施洗,因为那里水多。众人都去受洗。

那时约翰还没有下在监里。

约翰的门徒,和一个犹太人辩论洁净的礼。

就来见约翰说,拉比,从前同你在约旦河外,你所见证的那位,现在施洗,众人都往他那里去。

约翰说,若不是从天上赐的,人就不能得什么。

我曾说,我不是基督,是奉差遣在他前面的,你们可以给我作见证。

娶新妇的,就是新郎。新郎的朋友站着听见新郎的声音就甚喜乐。故此我这喜乐满足了。

他必兴旺,我必衰微。

从天上来的,是在万有之上。从地上来的,是属乎地,他所说的,也是属乎地从天上来的,是在万有之上。

他将所见所闻的见证出来,只是没有人领受他的见证。

那领受他见证的,就印上印,证明神是真的。

神所差来的,就说神的话。因为神赐圣灵给他,是没有限量的。

父爱子,已将万有交在他手里。

信子的人有永生。不信子的人得不着永生,(原文作不得见永生)神的震怒常在他身上。

\chapter{第4章}
主知道法利赛人听见他收门徒施洗比约翰还多,

(其实不是耶稣亲自施洗,乃是他的门徒施洗)

他就离了犹太,又往加利利去。

必须经过撒玛利亚。

于是到了撒玛利亚的一座城,名叫叙加,靠近雅各给他儿子约瑟的那块地。

在那里有雅各井。耶稣因走路困乏,就坐在井旁。那时约有午正。

有一个撒玛利亚的妇人来打水。耶稣对他说,请你给我水喝。

那时门徒进城买食物去了。

撒玛利亚的妇人对他说,你既是犹太人,怎吗向我一个撒玛列亚妇人要水喝呢。原来犹太人和撒玛利亚人没有来往。

耶稣回答说,你若知道神的恩赐,和对你说给我水喝的是谁,你必早求他,他也必早给了你活水。

妇人说,先生没有打水的器具,井又深,你从那里得活水呢。

我们的祖宗雅各,将这井留给我们。他自己和儿子并牲畜,也都喝这井里的水,难道你比他还大吗。

耶稣回答说,凡喝这水的,还要再渴。

人若喝我所赐的水就永远不渴。我所赐的水,要在他里头成为泉源,直涌到永生。

妇人说,先生,请把这水赐给我,叫我不渴,也不用来这吗远打水。

耶稣说,你去叫你丈夫也到这里来。

妇人说,我没有丈夫。耶稣说,你说没有丈夫,是不错的。

你已经有五个丈夫。你现在有的,并不是你的丈夫。你这话是真的。

妇人说,先生,我看出你是先知。

我们的祖宗在这山上礼拜。你们倒说,应当礼拜的地方是在耶路撒冷。

耶稣说,妇人,你当信我,时候将到,你们拜父,也不在这山上,也不在耶路撒冷。

你们所拜的,你们不知道。我们所拜的,我们知道。因为救恩是从犹太人出来的。

时候将到,如今就是了,那真正拜父的,

神是个灵(或无个字)所以拜他的,必须用心灵和诚实拜他。

妇人说,我知道弥赛亚,(就是那称为基督的)要来。他来了,必将一切的事都告诉我们。

耶稣说,这和你说话的就是他。

当下门徒回来,就希奇耶稣和一个妇人说话。只是没有人说,你是要什么。或说,你为什么和他说话。

那妇人就留下水罐子,往城里去,对众人说,

你们来看,有一个人将我素来所行的一切事,都给我说出来了,莫非这就是基督吗。

众人就出城往耶稣那里去。

这其间,门徒对耶稣说,拉比请吃。

耶稣说,我有食物吃,是你们不知道的。

门徒就彼此对问说,莫非有人拿什么给他吃吗。

耶稣说,我的食物,就是遵行差我来者的旨意,作成他的工。

你们岂不说,到收割的时候,还有四个月吗。我告诉你们,举目向田观看,庄稼已经熟了,(原文作发白)可以收割了。

收割的人得工价,积畜五谷到永生。叫撒种的和收割的一同快乐。

俗语说,那人撒种,这人收割,这话可见是真的。

我差你们去收你们所没有劳苦的。别人劳苦,你们享受他们所劳苦的。

那城里有好些撒玛利亚人信了耶稣,因为那妇人作见证说,他将我素来所行的一切事,都给我说出来了。

于是撒玛利亚人来见耶稣,求他在他们那里住下。他便在那里住了两天。

因耶稣的话,信的人就更多了。

便对妇人说,现在我们信,不是因为你的话,是我们亲自见了,知道这真是救世主。

过了那两天,耶稣离了那地方,往加利利去。

因为耶稣自己作见证说,先知在本地是没有人尊敬的。

到了加利利,加利利人既然看见他在耶路撒冷过节所行的一切事,就接待他。因为他们也是上去过节。

耶稣又到了加利利的迦拿,就是他从前变水为酒的地方。有一个大臣,他的儿子在迦百农患病。

他听见耶稣从犹太到了加利利,就来见他,求他下去医治他的儿子。因为他儿子快要死了。

耶稣就对他说,若不看见神迹奇事,你们总是不信。

那大臣说,先生,求你趁着我的孩子还没有死,就下去。

耶稣对他说,回去吧。你的儿子活了。那人信耶稣所说的话,就回去了。

正下去的时候,他的仆人迎见他,说,他的儿子活了。

他就问什么时候见好的。他们说,昨日未时热就退了。

他便知道这正是耶稣对他说,你的儿子活了的时候,他自己全家就都信了。

这是耶稣在加利利行的第二件神迹,是他从犹太回去以后行的。

\chapter{第5章}
这事以后,到了犹太的一个节期。耶稣就上耶路撒冷去。

在耶路撒冷,靠近羊门,有一个池子,希伯来话叫作毕士大,旁边有五个廊子。

里面躺着瞎眼的,瘸腿的,血气枯乾的,许多病人。(有古卷在此有等候水动

因为有天使按时下池搅动那水,水动之后,谁先下去,无论什么病,就痊愈了)。

在那里有一个人,病了三十八年。

耶稣看见他躺着,知道他病了许久,就问他说,你要痊愈吗。

病人回答说,先生,水动的时候,没有人把我放在池子里。我正去的时候,就有别人比我先下去。

耶稣对他说,起来,拿你的褥子走吧。

那人立刻痊愈,就拿起褥子走了。

那天是安息日,所以犹太人对那医好的人说,今天是安息日,你拿褥子是不可以的。

他却回答说,那使我痊愈的,对我说,拿你的褥子走吧。

他们问他说,对你说拿褥子走的,是什么人。

那医好的人不知道是谁。因为那里的人多,耶稣已经躲开了。

后来耶稣在殿里遇见他,对他说,你已经痊愈了。不要再犯罪。恐怕你遭遇的更加利害。

那人就去告诉犹太人,使他痊愈的是耶稣。

所以犹太人逼迫耶稣,因为他在安息日作了这事。

耶稣就对他们说,我父作事到如今,我也作事。

所以犹太人越发想要杀他。因他不但犯了安息日,并且称神为他的父,将自己和神当作平等。

耶稣对他们说,我实实在在的告诉你们,子凭着自己不能作什么,惟有看见父所作的,子才能作。父所作的事,子也照样作。

爱子,将自己所作的一切事指给他看。还要将比这更大的事指给他看,叫你们希奇。

父怎样叫死人起来,使他们活着,子也照样随自己的意思使人活着。

父不审判什么人,乃将审判的事全交与子。

叫人都尊敬子,如同尊敬父一样。不尊敬子的,就是不尊敬差子来的父。

我实实在在的告诉你们,那听我话,又信差我来者的,就有永生,不至于定罪,是已经出死入生了。

我实实在在的告诉你们,时候将到,现在就是了,死人要听见神儿子的声音。听见的人就要活了。

因为父怎样在自己有生命,就赐给他儿子也照样在自己有生命。

并且因为他是人子,就赐给他行审判的权柄。

你们不要把这事看作希奇。时候要到,凡在坟墓里的,都要听见他的声音,就出来。

行善的复活得生,作恶的复活定罪。

我凭着自己不能作什么。我怎吗听见,就怎吗审判。我的审判也是公平的。因为我不求自己的意思,只求那差我来者的意思。

我若为自己作见证,我的见证就不真。

另有一位给我作见证。我也知道他给我作的见证是真的。

你们曾差人到约翰那里,他为真理作过见证。

其实我所受的见证,不是从人来的。然而我说这些话,为要叫你们得救。

约翰是点着的明灯。你们情愿暂时喜欢他的光。

但我有比约翰更大的见证。因为父交给我要我成就的事,就是我所作的事,这便见证我是父所差来的。

差我来的父,也为我作过见证。你们从来没有听见他的声音,也没有看见他的形像。

你们并没有他的道存在心里。因为他所差来的,你们不信。

你们查考圣经。(或作应当查考圣经)因你们以为内中有永生。给我作见证的就是这经。

然而你们不肯到我这里来得生命。

我不受从人来的荣耀。

但我知道你们心里,没有神的爱。

我奉我父的名来,你们并不接待我。若有别人奉自己的名来,你们倒要接待他。

你们要互相受荣耀,却不求从独一之神来的荣耀,怎能信我呢。

不要想我在父面前要告你们。有一位告你们的,就是你们所仰赖的摩西。

你们如果信摩西,也必信我。因为他书上有指着我写的话。

你们若不信他的书,怎能信我的话呢。

\chapter{第6章}
这事以后,耶稣渡过加利利海,就是提比哩亚海。

有许多人,因为看见他在病人身上所行的神迹,就跟随他。

耶稣上了山,和门徒一同坐在那里。

那时犹太人的逾越节近了。

耶稣举目看见许多人来,就对腓力说,我们从那里买饼叫这些人吃呢。

他说这话,是要试验腓力。他自己原知道要怎样行。

腓力回答说,就是二十两银子的饼,叫他们各人吃一点,也是不够的。

有一个门徒,就是西门彼得的兄弟安得烈,对耶稣说,

在这里有一个孩童,带着五个大麦饼,两条鱼。只是分给这许多人,还算什么呢。

耶稣说,你们叫众人坐下。原来那地方的草多,众人就坐下。数目约有五千。

耶稣拿起饼来,祝谢了,就分给那坐着的人。分鱼也是这样,都随着他们所要的。

他们吃饱了,耶稣对门徒说,把剩下的零碎,收拾起来,免得有糟蹋的。

他们便将那五个大麦饼的零碎,就是众人剩下的,收拾起来,装满了十二个篮子。

众人看见耶稣所行的神迹。就说,这真是那要到世间来的先知。

耶稣既知道众人要来强逼他作王,就独自又退到山上去了。

到了晚上,他的门徒下海边去,

上了船,要过海往迦百农去。天已经黑了,耶稣还没有来到他们那里。

忽然狂风大作,海就翻腾起来。

门徒摇橹约行了十里多路,看见耶稣在海面上走,渐渐近了船,他们就害怕。

耶稣对他们说,是我。不要怕。

门徒就喜欢接他上船,船立时到了他们所要去的地方。

第二日,站在海那边的众人,知道那里没有别的船,只有一只小船,又知道耶稣没有同他的门徒上船,乃是门徒自己去的。

然而有几只小船从提比哩亚来,靠近主祝谢后分饼给人吃的地方。

众人见耶稣和门徒,都不在那里,就上了船,往迦百农去找耶稣。

既在海那边找着了,就对他说,拉比,是几时到这里来的。

耶稣回答说,我实实在在的告诉你们,你们找我,并不是因见了神迹,乃是因吃饼得饱。

不要为那必坏的食物劳力,要为那存到永生的食物劳力,就是人子要赐给你们的。因为人子是父神所印证的。

众人问他说,我们当行什么,才算作神的工呢。

耶稣回答说,信神所差来的,这就是作神的工。

他们又说,你行什么神迹,叫我们看见就信你。你到底作什么事呢。

我们的祖宗在旷野吃过吗哪,如经上写着说,他从天上赐下粮来给他们吃。

耶稣说,我实实在在的告诉你们,那从天上来的粮,不是摩西赐给你们的,乃是我父将天上的真粮赐给你们。

因为神的粮,就是那从天上降下来赐生命给世界的。

他们说,主阿,常将这粮赐给我们。

耶稣说,我就是生命的粮。到我这里来的,必定不饿。信我的,永远不渴。

只是我对你们说过,你们已经看见我,还是不信。

凡父所赐给我的人,必到我这里来。到我这里来的,我总不丢弃他。

因为我从天上!降下来,不是要按自己的意思行,乃是要按那差我来者的意思行。

差我来者的意思,就是他所赐给我的,叫一个也不失落,在末日却叫他复活。

因为我父的意思,是叫一切见子而信的人得永生。并且在末日我要叫他复活。

犹太人因为耶稣说,我是从天上降下来的粮,就私下议论他。

说,这不是约瑟的儿子耶稣吗。他的父母我们岂不认得吗。他如今怎吗说,我是从天上降下来的呢。

耶稣回答说,你们不要大家议论。

若不是差我来的父吸引人,就没有能到我这里来的。到我这里来的,在末日我要叫他复活。

在先知书上写着说,他们都要蒙神的教训。凡听见父之教训又学习的,就到我这里来。

这不是说,有人看过父,惟独从神来的,他看见过父。

我实实在在的告诉你们,信的人有永生。

我就是生命的粮。

你们的祖宗在旷野吃过吗哪,还是死了。

这是从天上降下来的粮,叫人吃了就不死。

我是从天上降下来生命的粮。人若吃这粮,就必永远活着。我所要赐的粮,就是我的肉,为世人之生命所赐的。

因此,犹太人彼此争论说,这个人怎能把他的肉,给我们吃呢。

耶稣说,我实实在在的告诉你们,你们若不吃人子的肉,不喝人子的血,就没有生命在你们里面。

吃我肉,喝我血的人就有永生。在末日我要叫他复活。

我的肉真是可吃的,我的血真是可喝的。

吃我肉喝我血的人,常在我里面,我也常在他里面。

永活的父怎样差我来,我又因父活着,照样,吃我肉的人,也要因我活着。

这就是从天上降下来的粮。吃这个粮的人,就永远活着,不像你们的祖宗吃过吗哪,还是死了。

这些话是耶稣在迦百农会堂里教训人说的。

他的门徒中有好些人听见了,就说,这话甚难,谁能听呢。

耶稣心里知道门徒为这话议论,就对他们说,这话是叫你们厌弃吗。(厌弃原文作跌倒)

倘或你们看见人子升到他原来所在之处,怎吗样呢。

叫人活着的乃是灵,肉体是无益的。我对你们所说的话,就是灵,就是生命。

只是你们中间有不信的人。耶稣从起头就知道,谁不信他,谁要卖他。

耶稣又说,所以我对你们说过,若不是蒙我父的恩赐,没有人能到我这里来。

从此他门徒中多有退去的,不再和他同行。

耶稣就对那十二个门徒说,你们也要去吗。

西门彼得回答说,主阿,你有永生之道,我们还归从谁呢。

我们已经信了,又知道你是神的圣者。

耶稣说,我不是拣选了你们十二个门徒吗,但你们中间有一个是魔鬼。

耶稣这话是指着加略人西门的儿子犹大说的。他本是十二个门徒里的一个,后来要卖耶稣的。

\chapter{第7章}
这事以后,耶稣在加利利游行,不愿在犹太游行。因为犹太人想要杀他。

当时犹太人的住棚节近了。

耶稣的弟兄就对他说,你离开这里上犹太去吧,叫你的门徒也看见你所行的事。

人要显扬名声,没有在暗处行事的。你如果行这事,就当将自己显明给世人看。

因为连他的弟兄说这话,是因为不信他。

耶稣就对他们说,我的时候还没有到。你们的时候常是方便的。

世人不能恨你们,却是恨我。因为我指证他们所作的事是恶的。

你们上去过节吧。我现在不上去过这节。因为我的时候还没有满。

耶稣说了这话,仍旧住在加利利。

但他兄弟上去以后,他也上去过节,不是明去,似乎是暗去的。

正在节期,犹太人寻梢耶稣说,他在那里。

众人为他纷纷议论。有的说,他是好人。有的说,不然,他是迷惑众人的。

只是没有人明明的讲论他,因为怕犹太人。

到了节期,耶稣上殿里去教训人。

犹太人就希奇说,这个人没有学过,怎吗明白书呢。

耶稣说,我的教训不是我自己的,乃是那差我来者的。

人若立志遵着他的旨意行,就必晓得这教训或是出于神,或是我凭着自己说的。

人凭着自己说,是求自己的荣耀。惟有求那差他来者的荣耀,这人是真的,在他心里没有不义。

摩西岂不是传律法给你们吗。你们却没有一个人守律法。为什么想要杀我呢。

众人回答说,你是被鬼附着了。谁想要杀你。

耶稣说,我作了一件事,你们都以为希奇。

摩西传割礼给你们,(其实不是从摩西起的,乃是从祖先起的)因此你们也要在安息日给人行割礼。

人若在安息日受割礼,免得违背摩西的律法。我在安息日叫一个人全然好了,你们就向我生气吗。

不可按外貌断定是非,总要按公平断定是非。

耶路撒冷人中有的说,这不是他们想要杀的人吗。

你看他还明明的讲道,他们也不向他说什么。难道官长真知道这是基督吗。

然而我们知道这个人从那里来。只是基督来的时候,没有人知道他是从那里来的。

那时耶稣在殿里教训人,大声说,你们也知道我,也知道我从那里来。我来并不是由于自己,但那差我来的是真的。你们不认识他。

我却认识他。因为我是从他来的,他也差了我来。

他们就想要捉拿耶稣。只是没有人下手,因为他的时候还没有到。

但众人中间有些信他的,说,基督来的时候,他所行的神迹,岂能比这人所行的更多吗。

法利赛人听见众人为耶稣这样纷纷议论,祭司长和法利赛人,就打发差役去捉拿他。

于是耶稣说,我还有不多的时候和你们同在,以后就回到差我来的那里去。

你们要找我,却找不着。我所在的地方你们不能到。

犹太人就彼此对问说,这人要往那里去,叫我们找不着呢。难道他要往散住希腊中的犹太人那里去教训希腊人吗。

他说,你们要找我,却找不着,我所在的地方,你们不能到。这话是什么意思呢。

节期的末日,就是最大之日,耶稣站着高声说,人若渴了,可以到我这里来喝。

信我的人,就如经上所说,从他复中要流出活水的江河来。

耶稣这话是指着信他之人,要受圣灵说的,那时还没有赐下圣灵来,因为耶稣尚未得着荣耀。

众人听见这话,有的说,这真是那先知。

有的说,这是基督。但也有的说,基督岂是从加利利出来的吗。

经上岂不是说,基督是大卫的后裔,从大卫本乡伯利恒出来的吗。

于是众人因着耶稣起了分争。

其中有人要捉拿他。只是无人下手。

差役回到祭司长和法利赛人那里。他们对差役说,你们为什么没有带他来呢。

差役回答说,从来没有像他这样说话的。

法利赛人说,你们也受了迷惑吗。

官长或是法利赛人,岂有信他的呢。

但这些不明白律法的百姓,是被咒诅的。

内中有尼哥底母,就是从前去见耶稣的,对他们说,

不先听本人的口供,不知道他所作的事,难道我们的律法还定他的罪吗。

他们回答说,你也是出于加利利吗。你且去查考,就可知道,加利利没有出过先知。

\chapter{第8章}
于是各人都回家去了。耶稣却往橄榄山去。

清早又回到殿里。众百姓都到他那里去,他就坐下教训他们。

文士和法利赛人,带着一个行淫时被拿的妇人来,叫他站在当中。

就对耶稣说,夫子,这妇人是正行淫之时被拿的。

摩西在律法上吩咐我们,把这样的妇人用石头打死。你说该把他怎吗样呢。

他们说这话,乃试探耶稣,要得着告他的把柄。耶稣却弯着腰用指头在地上画字。

他们还是不住的问他,耶稣直起腰来,对他们说,你们中间谁是没有罪的,谁就可以先拿石头打他。

于是又弯着腰用指头在地上画字。

他们听见这话,就从老到少一个一个的都出去了。只剩下耶稣一人。还有那妇人仍然站在当中。

耶稣就直起腰来,对他说,妇人,那些人在那里呢。没有人定你的罪吗。

他说,主阿,没有。耶稣说,我也不定你的罪。去吧。从此不要再犯罪了。

耶稣又对众人说,我是世上的光。跟从我的,就不在黑暗里走,必要得着生命的光。

法利赛人对他说,你是为自己作见证。你的见证不真。

耶稣说,我虽然为自己作见证,我的见证还是真的。因我知道我是从那里来,往那里去。你们却不知道我是从那里来,往那里去。

你们是以外貌(原文作凭肉身)判断人。我却不判断人。

就是判断人,我的判断也是真的。因为不是我独自在这里,还有差我来的父与我同在。

你们的律法上也记着说,两个人的见证是真的。

我是为自己作见证,还有差我来的父,也是为我作见证。

他们就问他说,你的父在那里。耶稣回答说,你们不认识我,也不认识我的父。若认识我,也就认识我的父。

这些话是耶稣在殿里的库房,教训人时所说的。也没有人拿他。因为他的时候还没有到。

耶稣又对他们说,我要去了,你们要找我,并且你们要死在罪中。我所去的地方,你们不能到。

犹太人说,他说我所去的地方,你们不能到,难道他要自尽吗。

耶稣对他们说,你们是从下头来的,我是从上头来的。你们是属这世界的,我不是属这世界的。

所以我对你们说,你们要死在罪中,你们若不信我是基督,必要死在罪中。

他们就问他说,你是谁。耶稣对他们说,就是我从起初所告诉你们的。

我有许多事讲论你们,判断你们,但那差我来的是真的。我在他那里所听见的,我就传给世人。

他们不明白耶稣是指着父说的。

所以耶稣说,你们举起人子以后,必知道我是基督,并且知道我没有一件事,是凭着自己作的。我说这话,乃是照着父所教训我的。

那差我来的,是与我同在。他没有撇下我独自在这里,因为我常作他所喜欢的事。

耶稣说这话的时候,就有许多人信他。

耶稣对信他的犹太人说,你们若常常遵守我的道,就真是我的门徒。

你们必晓得真理,真理必叫你们得以自由。

他们回答说,我们是亚伯拉罕的后裔,从来没有作过谁的奴仆。你怎吗说,你们必得以自由呢。

耶稣回答说,我实实在在的告诉你们。所有犯罪的,就是罪的奴仆。

奴仆不能永远住在家里,儿子是永远住在家里。

所以天父的儿子若叫你们自由,你们就真自由了。

我知道你们是亚伯拉罕的子孙,你们却想要杀我。因为你们心里容不下我的道。

我所说的,是在我父那里看见的。你们所行的,是在你们的父那里听见的。

他们说,我们的父就是亚伯拉罕。耶稣说,你们若是亚伯拉罕的儿子,就必行亚伯拉罕所行的事。

我将在神那里所听见的真理,告诉了你们,现在你们却想要杀我。这不是亚伯拉罕所行的事。

你们是行你们父所行的事。他们说,我们不是从淫乱生的。我们只有一位父就是神。

耶稣说,倘若神是你们的父,你们就必爱我。因为我本是出于神,也是从神而来,并不是由着自己来,乃是他差我来。

你们为什么不明白我的话呢,无非是因你们不能听我的道。

你们是出于你们的父魔鬼,你们父的私欲,你们偏要行,他从起初是杀人的,不守真里。因他心里没有真里,他说谎是出于自己,因他本来是说谎的,也是说谎之人的父。

我将真理告诉你们,你们就因此不信我。

你们中间谁能指证我有罪呢。我既然将真理告诉你们,为什么不信我呢。

出于神的,必听神的话。你们不听,因为你们不是出于神。

犹太人回答说,我们说,你是撒玛利亚人,并且是鬼附着的,这话岂不正对吗。

耶稣说,我不是鬼附着的。我尊敬我的父,你们倒轻慢我。

我不求自己的荣耀。有一位为我求荣耀定是非的。

我实实在在的告诉你们,人若遵守我的道,就永远不见死。

犹太人对他说,现在我们知道你是鬼附着的。亚伯拉罕死了,众先知也死了。你还说,人若遵守我的道,就永远不尝死味。

难道你比我们的祖宗亚伯拉罕还大吗。他死了,众先知也死了。你将自己当作什么人呢。

耶稣回答说,我若荣耀自己,我的荣耀就算不得什么。荣耀我的乃是我的父,就是你们所说是你们的神。

你们未曾认识他。我却认识他。我若说不认识他,我就是说谎的,像你们一样,但我认识他,也遵守他的道。

你们的祖宗亚伯拉罕欢欢喜喜的仰望我的日子。既看见了,就快乐。

犹太人说,你还没有五十岁,岂能见过亚伯拉罕呢。

耶稣说,我实实在在的告诉你们,还没有亚伯拉罕,就有了我。

于是他们拿石头要打他。耶稣却躲藏,从殿里出去了。

\chapter{第9章}
耶稣过去的时候,看见一个人生来是瞎眼的。

门徒问耶稣说,拉比,这人生来是瞎眼的,是谁犯了罪,是这人呢,是他父母呢。

耶稣回答说,也不是这人犯了罪,也不是他父母犯了罪,是要在他身上显出神的作为来。

趁着白日,我们必须作那差我来者的工。黑夜将到,就没有人能作工了。

我在世上的时候,是世上的光。

耶稣说了这话,就吐唾沫在地上,用唾沫和泥抹在瞎子的眼睛上,

对他说,你往西罗亚池子里去洗,(西罗亚翻出来,就是奉差遣)他去一洗,回头就看见了。

他的邻舍,和那素常见他是讨饭的,就说,这不是那从前坐着讨饭的人吗。

有人说,是他。又有人说,不是,却是像他。他自己说,是我。

他们对他说,你的眼睛是怎吗开的呢。

他回答说,有一个人名叫耶稣。他和泥抹我的眼睛,对我说,你往西罗亚池子去洗。我去一洗,就看见了。

他们说,那个人在那里。他说,我不知道。

他们把从前瞎眼的人,带到法利赛人那里。

耶稣和泥开他眼睛的日子是安息日。

法利赛人也问他是怎吗得看见的。瞎子对他们说,他把泥抹在我的眼睛上,我去一洗,就看见了。

法利赛人中有的说,这个人不是从神来的,因为他不守安息日。又有人说,一个罪人怎能行这样的神迹呢。他们就起了分争。

他们又对瞎子说,他既然开了你的眼睛,你说他是怎样的人呢。他说,是个先知。

犹太人不信他从前是瞎眼,后来能看见的,等到叫了他的父母来,

问他们说,这是你们的儿子吗。你们说他生来是瞎眼的,如今怎吗能看见了呢。

他父母回答说,他是我们的儿子,生来就瞎眼,这是我们知道的。

至于他如今怎吗能看见,我们却不知道。是谁开了他的眼睛,我们也不知道。他已经成了人,你们问他吧。他自己必能说。

他父母说这话,是怕犹太人,因为犹太人已经商议定了,若有认耶稣是基督的,要把他赶出会堂。

因此他父母说,他已经成了人,你们问他吧。

所以法利赛人第二次叫了那从前瞎眼的人来,对他说,你该将荣耀归给神。我们知道这人是个罪人。

他说,他是个罪人不是,我不知道。有一件事我知道,从前我是眼瞎的,如今能看见了。

他们就问他说,他向你作什么,是怎吗开了你的眼睛呢。

他回答说,我方才告诉你们,你们不听。为什么又要听呢。莫非你们也要作他的门徒吗。

他们就骂他说,你是他的门徒。我们是摩西的门徒。

神对摩西说话,是我们知道的。只是这个人,我们不知道他从那里来。

那人回答说,他开了我的眼睛,你们竟不知道他从那里来,这真是奇怪。

我们知道神不听罪人。惟有敬奉神旨意的,神才听他。

从创世以来,未曾听见有人把生来是瞎子的眼睛开了。

这人若不是从神来的,什么也不能作。

他们回答说,你全然生在罪孽中,还要教训我们吗。于是把他赶出去了。

耶稣听说他们把他赶出去。后来遇见他,就说,你信神的儿子吗。

他回答说,主阿,谁是神的儿子,叫我信他呢。

耶稣说,你已经看见他,现在和你说话的就是他。

他说,主阿,我信。就拜耶稣。

耶稣说,我为审判到这世上来,叫不能看见的,可以看见。能看见的,反瞎了眼。

同他在那里的法利赛人,听见这话,就说,难道我们也瞎了眼吗。

耶稣对他们说,你们若瞎了眼,就没有罪了。但如今你们说,我们能看见,所以你们的罪还在。

\chapter{第10章}
我实实在在的告诉你们,人进羊圈,不从门进去,倒从别处爬进去,那人就是强盗。

从门进去的,才是羊的牧人。

看门的就给他开门。羊也听他的声音。他按着名叫自己的羊,把羊领出来。

既放出自己的羊来,就在前头走,羊也跟着他,因为认得他的声音。

羊不跟着生人,因为不认得他的声音。必要逃跑。

耶稣将这比喻告诉他们。但他们不明白所说的是什么意思。

所以耶稣又对他们说,我实实在在的告诉你们,我就是羊的门,

凡在我以先来的,都是贼,是强盗。羊却不听他们。

我就是门。凡从我进来的,必然得救,并且出入得草吃。

盗贼来,无非要偷窃,杀害,毁坏。我来了,是要叫羊(或作人)得生命,并且得的更丰盛。

我是好牧人,好牧人为羊舍命。

若是雇工,不是牧人,羊也不是他自己的,他看见狼来,就撇下羊逃走。狼抓住羊,赶散了羊群。

雇工逃走,因为他是雇工,并不顾念羊。

我是好牧人。我认识我的羊,我的羊也认识我。

正如父认识我,我也认识父一样。并且我为羊舍命。

我另外有羊,不是这个圈里的。我必须领他们来,他们也要听我的声音。并且要合成一群,归一个牧人了。

我父爱我,因我将命舍去,好再取回来。

没有人夺我的命去,是我自己舍的。我有权柄舍了,也有权柄取回来。这是我从父所受的命令。

犹太人为这些话,又起了分争。

内中有好些人说,他是被鬼附着,而且疯了。为什么听他呢。

又有人说,这不是鬼附之人所说的话。鬼岂能叫瞎子的眼睛开了呢。

在耶路撒冷有修殿节。是冬天的时候。

耶稣在殿里所罗门的廊下行走。

犹太人围着他,说,你叫我们犹疑不定到几时呢。你若是基督,就明明的告诉我们。

耶稣回答说,我已经告诉你们,你们不信。我奉我父之名所行的事,可以为我作见证。

只是你们不信,因为你们不是我的羊。

我的羊听我的声音,我也认识他们,他们也跟着我。

我又赐给他们永生。他们永不灭亡,谁也不能从我手里把他们夺去。

我父把羊赐给我,他比万有都大。谁也不能从我父手里把他夺去。

我与父原为一。

犹太人又拿起石头要打他。

耶稣对他们说,我从父显出许多善事给你们看,你们是为那一件事拿石头打我呢。

犹太人回答说,我们不是为善事拿石头打你,是你说僭妄的话。又为你是个人,反将自己当作神。

耶稣说,你们的律法上岂不是写着,我曾说你们是神吗。

经上得话是不能废的。若那些承受神道的人,尚且称为神,

父所分别为圣,又差到世间来的,他自称是神的儿子,你们还向他说,你说僭妄的话吗。

我若不行我父的事,你们就不必信我。

我若行了,你们纵然不信我,也当信这些事。叫你们又知道,又明白,父在我里面,我也在父里面。

他们又要拿他。他却逃出他们的手走了。

耶稣又往约旦河外去,到了约翰起初施洗的地方,就住在那里。

有许多人来到他那里。他们说约翰一件神迹没有行过。但约翰指着这人所说的一切话都是真的。

在那里信耶稣的人就多了。

\chapter{第11章}
有一个患病的人,名叫拉撒路,住在伯大尼,就是马利亚和他姐姐马大的村庄。

这马利亚就是那用香膏抹主,又用头发擦他脚的。患病的拉撒路是他的兄弟。

他姊妹两个就打发人去见耶稣说,主阿,你所爱的人病了。

耶稣听见就说,这病不至于死,乃是为神的荣耀,叫神的儿子因此得荣耀。

耶稣素来爱马大,和他妹子,并拉撒路。

听见拉撒路病了,就在所居之地,仍住了两天。

然后对门徒说,我们再往犹太去吧。

门徒说,拉比,犹太人近来要拿石头打你,你还往那里去吗。

耶稣回答说,白日不是有十二小时吗。人在白日走路,就不至跌倒,因为看见这世上的光。

若在黑夜走路,就必跌倒,因为他没有光。

耶稣说了这话,随后对他们说,我们的朋友拉撒路睡了,我去叫醒他。

门徒说,主阿,他若睡了,就必好了。

耶稣这话是指着他死说的。他们却以为是说照常睡了。

耶稣就明明的告诉他们说,拉撒路死了。

我没有在那里就欢喜,这是为你们的缘故,好叫你们相信。如今我们可以往他那里去吧。

多马,又称为低土马,就对那同作门徒的说,我们也去和他同死吧。

耶稣到了,就知道拉撒路在坟墓里,已经四天了。

伯大尼离耶路撒冷不远,约有六里路。

有些犹太人来看马大和马利亚,要为他们的兄弟安慰他们。

马大听见耶稣来了,就出去迎接他。马利亚却仍然坐在家里。

马大对耶稣说,主阿,你若早在这里,我兄弟必不死。

就是现在,我也知道,你无论向神求什么,神也必赐给你。

耶稣说,你兄弟必然复活。

马大说,我知道在末世复活的时候,他必复活。

耶稣对他说,复活在我,生命也在我。信我的人,虽然死了,也必复活。

凡活着信我的人,必永远不死。你信这话吗。

马大说,主阿,是的,我信你是基督,是神的儿子,就是那要临到世界的。

马大说了这话,就回去暗暗的叫他妹子,马利亚说,夫子来了,叫你。

马利亚听见了就急忙起来,到耶稣那里去。

那时,耶稣还没有进村子,仍在马大迎接他的地方。

那些同马利亚在家里安慰他的犹太人,见他急忙起来出去,就跟着他,以为他要往坟墓那里胎哭。

马利亚到了耶稣那里,看见他,就俯伏在他脚前,说,主阿,你若早在这里,我兄弟必不死。

耶稣看见他哭,并看见与他同来的犹太人也哭,就心里悲叹,又甚忧愁。

便说,你们把他安放在那里。他们回答说,请主来看。

耶稣哭了。

犹太人就说,你看他爱这人是何等恳切。

其中有人说,他既然开了瞎子的眼睛,岂不能叫这人不死吗。

耶稣又心里悲叹,来到坟墓前。那坟墓是个洞,有一块石头挡着。

耶稣说,你们把石头挪开。那死人的姐姐马大对他说,主阿,他现在必是发臭了,因为他死了已经四天了。

耶稣说,我不是对你说过,你若信,就必看见神的荣耀吗。

他们就把石头挪开。耶稣举目望天说,父阿,我感谢你,因为你已经听我。

我也知道你常听我,但我说这话,是为周围站着的众人,叫他们信是你差了我来。

说了这话,就大声呼叫说,拉撒路出来。

那死人就出来了,手脚裹着布,脸上包着手巾。耶稣对他们说,解开,叫他走。

那些来看马利亚的犹太人,见了耶稣所作的事,就多有信他的。

但其中也有去见法利赛人的,将耶稣所作的事告诉他们。

祭司长和法利赛人聚集公会,说,这人行好些神迹,我们怎吗办呢。

若这样由着他,人人都要信他。罗马人也要来夺我们的地土,和我们的百姓。

内中有一个人,名叫该亚法,本年作大祭司,对他们说,你们不知道什么。

独不想一个人替百姓死,免得通国灭亡,就是你们的益处。

他这话不是出于自己,是因他本年作大祭司,所以预言耶稣将要替这一国死。

也不但替这一国死,并要将神四散的子民,都聚集归一。

从那日起,他们就商议要杀耶稣。

所以耶稣不再显然行在犹太人中间,就离开那里往靠近矿野的地方。到了一座城,名叫以法莲,就在那里和门徒同住。

犹太人的逾越节近了。有许多人从乡下上耶路撒冷去,要在节前洁净自己。

他们就寻梢耶稣,站在殿里彼此说,你们的意思如何,他不来过节吗。

那时,祭司长和法利赛人早已吩咐说,若有人知道耶稣在那里,就要报明,好去拿他。

\chapter{第12章}
逾越节前六日,耶稣来到伯大尼,就是他叫拉撒路从死里复活之处。

有人在那里给耶稣豫备筵席。马大伺候,拉撒路也在那同耶稣坐席的人中。

马利亚就拿着一斤极贵的真哪哒香膏,抹耶稣的脚,又用自己的头发去擦。屋里就满了膏的香气。

有一个门徒,就是那将要卖耶稣的加略人犹大,

说,这香膏为什么不卖三十两银子周济穷人呢。

他说这话,并不是挂念穷人,乃因他是个贼,又带着钱曩,常取其中所存的。

耶稣说,由他吧,他是为我安葬之日存留的。

因为常有穷人和你们同在。只是你们不常有我。

有许多犹太人知道耶稣在那里,就来了,不但是为耶稣的缘故,也是要看他从死里所复活的拉撒路。

但祭司长商议连拉撒路也要杀了。

因有好些犹太人,为拉撒路的缘故,回去信了耶稣。

第二天,有许多上来过节的人,听见耶稣将到耶路撒冷,

就拿着棕树枝,出去迎接他,喊着说,和撒那,奉主名来的以色列王,是应当称颂的。

耶稣得了一个驴驹,就骑上。如经上所记的说,

锡安的民哪,(民原文作女子)不要惧怕,你的王骑着驴驹来了。

这些事门徒起先不明白。等到耶稣得了荣耀以后,才想起这话是指着他写的,并且众人果然向他这样行了。

当耶稣呼唤拉撒路叫他从死复活出坟墓的时候,同耶稣在那里的众人,就作见证。

众人因听见耶稣行了这神迹,就去迎接他。

法利赛人彼此说,看哪,你们是徒劳无益,世人都随从他去了。

那时,上来过节礼拜的人中,有几个希腊人。

他们来见加利利伯赛大的腓力,求他说,先生,我们愿意见耶稣。

腓力去告诉安得烈,安得烈同腓力去告诉耶稣。

耶稣说,人子得荣耀的时候到了。

我实实在在的告诉你们,一粒麦子不落在地里死了,仍旧是一粒。若死了,就结出许多子粒来。

爱惜自己生命的,就失丧生命。在这世上恨恶自己生命的,就要保守生命到永生。

若有人服事我,就当跟从我。我在那里,服事我的人,也要在那里。若有人服事我,我父必尊重他。

我现在心里忧愁,我说什么才好呢。父阿,救我脱离这时候。但我原是为这时候来的。

父阿,愿你荣耀你的名。当时就有声音从天上来说,我已经荣耀了我的名,还要再荣耀。

站在旁边的众人听见,就说,打雷了。还有人说,有天使对他说话。

耶稣说,这声音不是为我,是为你们来的。

现在这世界受审判。这世界的王要被赶出去。

我若从地上被举起来,就要吸引万人来归我。

耶稣这话原是指着自己将要怎样死说的。

众人回答说,我们听见律法上有话说,基督是永存的。你怎吗说,人子必须被举起来呢。这人子是谁呢。

耶稣对他们说,光在你们中间,还有不多的时候,应当趁着有光行走,免得黑暗临到你们。那在黑暗里行走的,不知道往何处去。

你们应当趁着有光,信从这光,使你们成为光明之子。

耶稣说了这话,就离开他们,隐藏了。

他虽然在他们面前行了许多神迹,他们还是不信他。

这是要应验先知以赛亚的话说,主阿,我们所传的,有谁信呢。

他们所以不能信,因为以赛亚又说,主叫他们瞎了眼,硬了心,免得他们眼睛看见,心里明白,回转过来,我就医治他们。

以赛亚因为看见他的荣耀,就指着他说这话。

虽然如此,官长中却有好些信他的。只因法利赛人的缘故,就不承认,恐怕被赶出会堂。

这是因他们爱人的荣耀,过于爱神的荣耀。

耶稣大声说,信我的,不是信我,乃是信那差我来的。

人看见我,就是看见那差我来的。

我到世上来,乃是光,叫凡信我的,不住在黑暗里。

若有人听见我的话不遵守,我不审判他。我来本不是要审判世界,乃是要拯救世界。

弃绝我不领受我话的人,有审判他的。就是我所讲的道,在末日要审判他。

因为我没有凭着自己讲。惟有差我来的父,已经给我命令,叫我说什么,讲什么。

我也知道他的命令就是永生。故此我所讲的话,正是照着父对我所说的。

\chapter{第13章}
逾越节以前,耶稣知道自己离世归父的时候到了。他既爱世间属自己的人,就爱他们到底。

吃晚饭的时候,(魔鬼已将卖耶稣的意思,放在西门的儿子加略人犹大心里)。

耶稣知道父已将万有交在他手里,且知道自己是从神出来的,又要归到神那里去,

就离席站起来脱了衣服,拿一条手巾束腰。

随后把水倒在盆里,就洗门徒的脚,并用自己所束的手巾擦乾。

挨到西门彼得,彼得对他说,主阿,你洗我的脚吗。

耶稣回答说,我所作的,你如今不知道,后来必明白。

彼得说,你永不可洗我的脚。耶稣说,我若不洗你,你就与我无分了。

西门彼得说,主阿,不但我的脚,连手和头也要洗。

耶稣说,凡洗过澡的人,只要把脚一洗,全身就乾净了。你们是乾净的,然而不都是乾净的。

耶稣原知道要卖他的是谁,所以说,你们不都是乾净的。

耶稣洗完了他们的脚,就穿上衣服,又坐下,对他们说,我向你们所作的,你们明白吗。

你们称呼我夫子,称呼我主,你们说的不错。我本来是。

我是你们的主,你们的夫子,尚且洗你们的脚,你们也当彼此洗脚。

我给你们作了榜样,叫你们照着我向你们所作的去作。

我实实在在的告诉你们,仆人不能大于主人。差人也不能大于差他的人。

你们既知道这事,若是去行就有福了。

我这话不是指着你们众人说的。我知道我所拣选的是谁。现在要应验经上的话,说,同我吃饭的人,用脚踢我。

如今事情还没有成就,我要先告诉你们,叫你们到事情成就的时候,可以信我是基督

我实实在在的告诉你们,有人接待我所差遣的,就是接待我。接待我,就是接待那差遣我的。

耶稣说了这话,心里忧愁,就明说,我实实在在的告诉你们,你们中间有一个人要卖我了。

门徒彼此对看,猜不透所说的是谁。

有一个门徒,是耶稣所爱的,侧身挨近耶稣的怀里。

西门彼得点头对他说,你告诉我们,主是指着谁说的。

那门徒便就势靠着耶稣的胸膛,问他说,主阿,是谁呢。

耶稣回答说,我蘸一点饼给谁,就是谁。耶稣就蘸了一点饼,递给加略人西门的儿子犹大。

他吃了以后,撒但就入了他的心。耶稣对他说,你所作的快作吧。

同席的人,没有一个知道是为什么对他们说这话。

有人因犹大带着钱曩,以为耶稣是对他说,你去买我们过节所应用的东西。或是叫他拿什么周济穷人。

犹大受了那点饼,立刻就出去。那时候是夜间了。

他既出去,耶稣就说,如今人子得了荣耀,神在人子身上也得了荣耀。

神要因自己荣耀人子,并且要快快的荣耀他。

小子们,我还有不多的时候,与你们同在。后来你们要找我,但我所去的地方,你们不能到。这话我曾对犹太人说过,如今也照样对你们说。

我赐给你们一条新命令,乃是叫你们彼此相爱。我怎样爱你们,你们也要怎样相爱。

你们若有彼此相爱的心,众人因此就认出你们是我的门徒了。

西门彼得问耶稣说,主往那里去。耶稣回答说,我所去的地方,你现在不能跟我去。后来却要跟我去。

彼得说,主阿,我为什么现在不能跟你去。我愿意为你舍命。

耶稣说,你愿意为我舍命吗。我实实在在的告诉你,鸡叫以先,你要三次不认我。

\chapter{第14章}
你们心里不要忧愁。你们信神,也当信我。

在我父的家里,有许多住处。若是没有,我就早已告诉你们了。我去原是为你们预备地方去。

我若去为你们预备了地方,就必再来接你们到我那里去,我在那里,叫你们也在那里。

我往那里去,你们知道。那条路,你们也知道。(有古卷作我往那里去你们知道那条路)

多马对他说,主阿,我们不知道你往那里去,怎吗知道那条路呢。

耶稣说,我就是道路,真理,生命。若不藉着我。没有人能到父那里去。

你们若认识我,也就认识我的父。从今以后,你们认识他,并且已经看见他。

腓力对他说,求主将父显给我们看,我们就知足了。

耶稣对他说,腓力,我与你们同在这样长久,你还不认识我吗。人看见了我,就是看见了父。你怎吗说,将父显给我们看呢。

我在父里面,父在我里面,你不信吗。我对你们所说的话,不是凭着自己说的,乃是住在我里面的父作他自己的事。

你们当信我,我在父里面,父在我里面。既或不信,也当因我所作的事信我。

我实实在在的告诉你们,我所作的事,信我的人也要作。并且要作比这更大的事。因为我往父那里去。

你们奉我的名,无论求什么,我必成就,叫父因子得荣耀。

你们若奉我的名求什么,我必成就。

你们若爱我,就必遵守我的命令。

我要求父,父就另外赐给你们一位保惠师,(或作训慰师下同)叫他永远与你们同在。

就是真理的圣灵,乃世人不能接受的。因为不见他,也不认识他。你们却认识他。因他常与你们同在,也要在你们里面。

我不撇下你们为孤儿,我必到你们这里来。

还有不多的时候,世人不再看见我。你们却看见我。因为我活着,你们也活着。

到那日,你们就知道我在父里面,你们在我里面,我也在你们里面。

有了我的命令又遵守的,这人就是爱我的。爱我的必蒙我父爱他,我也要爱他,并且要向他显现。

犹大(不是加略人犹大)问耶稣说,主阿,为什么要向我们显现,不向世人显现呢。

耶稣回答说,人若爱我,就必遵守我的道。我父也必爱他,并且我们要到他那里去,与他同住。

不爱我的人就不遵守我的道。你们所听见的道不是我的,乃是差我来之父的道。

我还与你们同住的时候,已将这些话对你们说了。

但保惠师,就是父因我的名所要差来的圣灵,他要将一切的事,指教你们,并且要叫你们想起我对你们所说的一切话。

我留下平安给你们,我将我的平安赐给你们。我所赐的,不像世人所赐的。你们心里不要忧愁,也不要胆怯。

你们听见我对你们说了,我去还要到你们这里来。你们若爱我,因我到父那里去,就必喜乐,因为父是比我大的。

现在事情还没有成就,我豫先告诉你们,叫你们到事情成就的时候,就可以信。

以后我不再和你们多说话,因为这世界的王将到。他在我里面是毫无所有。

但要叫世人知道我爱父,并且父怎样吩咐我,我就怎样行。起来我们走吧。

\chapter{第15章}
我是真葡萄树,我父是栽培的人。

凡属我不结果子的枝子,他就剪去。凡结果子的,

现在你们因我讲给你们的道,已经乾净了。

你们要常在我里面,我也常在你们里面。枝子若不常在葡萄树上,自己就不能结果子。你们若不常在我里面,也是这样。

我是葡萄树,你们是枝子。常在我里面的,我也常在他里面,这人就多结果子。因为离了我,你们就不能作什么。

人若不常在我里面,就像枝子丢在外面枯乾,人拾起来,扔在火里烧了。

你们若常在我里面,我的话也常在你们里面,凡你们所愿意的,祈求就给你们成就。

你们多结果子,我父就因此得荣耀,你们也就是我的门徒了。

我爱你们,正如父爱我一样。你们要常在我的爱里。

你们若遵守我的命令,就常在我的爱里。正如我遵守了我父的命令,常在他的爱里。

这些事我已经对你们说了,是要叫我的喜乐,存在你们心里,并叫你们的喜乐可以满足。

你们要彼此相爱,像我爱你们一样,这就是我的命令。

人为朋友舍命,人的爱心没有比这个更大的。

你们若遵行我所吩付的,就是我的朋友了。

以后我不再称你们为仆人。因仆人不知道主人所作的事。我乃称你们为朋友。因我从我父所听见的。已经都告诉你们了。

不是你们拣选了我,是我拣选了你们,并且分派你们去结果子,叫你们的果子长存。使你们奉我的名,无论向父求什么,他就赐给你们。

我这样吩咐你们,是要叫你们彼此相爱。

世人若恨你们,你们知道(或作该知道)恨你们以先,已经恨我了。

你们若属世界,世界必爱属自己的。只因你们不属世界。乃是我从世界拣选了你们,所以世界就恨你们。

你们要记念我从前对你们所说的话,仆人不能大于主人。他们若逼迫了我,也要逼迫你们。若遵守了我的话,也要遵守你们的话。

但他们因我的名,要向你们行这一切的事,因为他们不认识那差我来的。

我若没有来教训他们,他们就没有罪。但如今他们的罪无可推诿了。

恨我的,也恨我的父。

我若没有在他们中间行过别人未曾行的事,他们就没有罪。但如今连我与我的父,他们也看见也恨恶了。

这要应验他们律法上所写的话说,他们无故恨我。

但我要从父那里差保惠师来,就是从父出来真里的圣灵。他来了,就要为我作见证。

你们也要作见证,因为你们从起头就与我同在。

\chapter{第16章}
我已将这些事告诉你们,使你们不至于跌倒。

人要把你们赶出会堂。并且时候将到,凡杀你们的,就以为是事奉神。

他们这样行,是因未曾认识父,也未曾认识我。

我将这些事告诉你们,是叫你们到了时候,可以想起我对你们说过了。我起先没有将这事告诉你们,因为我与你们同在。

现今我往差我来的父那里去。你们中间并没有人问我,你往那里去。

只因我将这事情告诉你们,你们就满心忧愁。

然而我将真情告诉你们。我去是与你们有益的。我若不去,保惠师就不到你们这里来。我若去,就差他来。

他既来了,就要叫世人为罪,为义,为审判,自己责备自己。

为罪,是因他们不信我。

为义,是因我往父那里去,你们就不再见我。

为审判,是因这世界的王受了审判。

我还有好些事要告诉你们,但你们现在担当不了(或作不能领会)。

只等真理的圣灵来了,他要引导你们明白(原文作进入)一切的真理。因为他不是凭自己说的,乃是把他所听见的都说出来。并要把将来的事告诉你们。

他要荣耀我。因为他要将受于我的,告诉你们。

凡父所有的,都是我的,所以我说,他要将受于我的,告诉你们。

等不多时,你们就不得见我。再等不多时,你们还要见我。

有几个门徒就彼此说,他对我们说,等不多时,你们就不得见我。再等不多时,你们还要见我。又说,因我往父那里去。这是什么意思呢。

门徒彼此说,他说等不多时,到底是什么意思呢。我们不明白他所说的话。

耶稣看出他们要问他,就说,我说等不多时,你们就不得见我,再等不多时,你们还要见我。你们为这话彼此相问吗。

我实实在在的告诉你们,你们将要痛哭,哀号,世人倒要喜乐。你们将要忧愁,然而你们的忧愁,要变为喜乐。

妇人生产的时候,就忧愁,因为他的时候到了。既生了孩子,就不再记念那苦楚,因为欢喜世上生了一个人。

你们现在也是忧愁。但我要再见你们,你们的心就喜乐了。这喜乐,也没有人能夺去。

到那日,你们什么也就不问我了。我实实在在的告诉你们,你们若向父求什么,他必因我的名,赐给你们。

向来你们没有奉我的名求什么,如今你们求就必得着,叫你们的喜乐可以满足。

这些事,我是用比喻对你们说的。时候将到,我不再用比喻对你们说,乃要将父明明的告诉你们。

到那日,你们要奉我的名祈求。我并不对你们说,我要为你们求父。

父自己爱你们,因为你们已经爱我,又信我是从父出来的。

我从父出来,到了世界。我又离开世界,往父那里去。

门徒说,如今你是明说,并不用比喻了。

现在我们晓得你凡事都知道,也不用人问你。因此我们信你是从神出来的。

耶稣说,现在你们信吗。

看哪,时候将到,且是已经到了,你们要分散,各归自己的地方去,留下我独自一人。其实我不是独自一人,因为有父与我同在。

我将这事告诉你们,是叫你们在我里面有平安。在世上你们有苦难。但你们可以放心,我已经胜了世界。

\chapter{第17章}
耶稣说了这话,就举目望天说,父阿,时候到了。愿你荣耀你的儿子,使儿子也荣耀你。

正如你曾赐给他权柄,管理凡有血气的,叫他将永生赐给你所赐给他的人。

认识你独一的真神,并且认识你所差来的耶稣基督,这就是永生。

我在地上已经荣耀你,你所托付我的事,我已成全了。

父阿,现在求你使我同你享荣耀,就是未有世界以先,我同你所有的荣耀。

你从世上赐给我的人,我已将你的名显明与他们。他们本是你的,你将他们赐给我,他们也遵守了你的道。

如今他们知道,凡你所赐给我的,都是从你那里来的。

因为你所赐给我的道,我已经赐给他们。他们也领受了,又确实知道,我是从你出来的,并且信你差了我来。

我为他们祈求。不为世人祈求,却为你所赐给我的人祈求,因他们本是你的。

凡是我的都是你的,你的也是我的。并且我因他们得了荣耀。

从今以后,我不在世上,他们却在世上,我往你那里去。圣父阿,求你因你所赐给我的名保守他们,叫他们合而为一,像我们一样。

我与他们同在的时候,因你所赐给我的名,保守了他们,我也护卫了他们,其中除了那灭亡之子,没有一个灭亡的。好叫经上的话得应验。

现在我往你们那里去。我还在世上说这话,是叫他们心里充满我的喜乐。

我已将你的道赐给他们。世界又恨他们,因为他们不属世界,正如我不属世界一样。

我不求你叫他们离开世界,只求你保守他们脱离

他们不属世界,正如我不属世界一样。

求你用真理使他们成圣。你的道就是真理。

你怎样差我到世上,我也照样差他们到世上。

我为他们的缘故,自己分别为圣,叫他们也因真理成圣。

我不但为这些人祈求,也为那些因他们的话信我的人祈求。

使他们都合而为一。正如你父在我里面,我在你里面,使他们也在我们里面。叫世人可以信你差了我来。

你赐给我的荣耀,我已经赐给他们,使他们合而为一,像我们合而为一。

我在他们里面,你在我里面,使他们完完全全的合而为一。叫世人知道你差了我来,也知道你爱他们如同爱我一样。

父阿,我在那里,愿你所赐给我的人也同我在那里,叫他们看见你所赐给我的荣耀。因为创立世界以前,你已经爱我了。

公义的父阿,世人未曾认识你,我却认识你。这些人也知道你差了我来。

我已将你的名指示他们,还要指示他们,使你所爱我的爱在他们里面,我也在他们里面。

\chapter{第18章}
耶稣说了这话,就同门徒出去,过了汲沦溪,在那里有一个园子,他和门徒进去了。

卖耶稣的犹大也知道那地方。因为耶稣和门徒屡次上那里去聚集。

犹大领了一队兵,和祭司长并法利赛人的差役,拿着灯笼,火把,兵器,就来到园里。

耶稣知道将要临到自己的一切事,就出来,对他们说,你们找谁。

他们回答说,找拿撒勒人耶稣。耶稣说,我就是。卖他的犹大也同他们站在那里。

耶稣一说我就是,他们就退后倒在地上。

他又问他们说,你们找谁。他们说,找拿撒勒人耶稣。

耶稣说,我已经告诉你们,我就是。你们若找我,就让这些人去吧。

这要应验耶稣从前的话,说,你所赐给我的人,我没有失落一个。

西门彼得带着一把刀,就拔出来,将大祭司的仆人砍了一刀,削掉他的右耳。那仆人名叫马勒古。

耶稣就对彼得说,收刀入鞘吧。我父所给我的那杯,我岂可不喝呢。

那队兵和千夫长并犹太人的差役,就拿住耶稣,把他捆绑了。

先带到亚那面前。因为亚那是本年作大祭司该亚法的岳父。

这该亚法,就是从前向犹太人发议论说,一个人替百姓死是有益的那位。

西门彼得跟着耶稣,还有一个门徒跟着。那门徒是大祭司所认识的。他就同耶稣进了大祭司的院子。

彼得却站在门外。大祭司所认识的那门徒出来,和看门使女说了一声,就领彼得进去。

那看门的使女对彼得说,你不也是这人的门徒吗。他说,我不是。

仆人和差役,因为天冷,就生了炭火,站在那里烤火。彼得也同他们站着烤火。

大祭司就以耶稣的门徒和他的教训盘问他。

耶稣回答说,我从来是明明的对世人说话。我常在会堂和殿里,就是犹太人聚集的地方,教训人。我在暗地里,并没有说什么。

你为什么问我呢。可以问那听见的人,我对他们说的是什么。我所说的,他们都知道。

耶稣说了这话,旁边站着的一个差役,用手掌打他说,你这样回答大祭司吗。

耶稣说,我若说的不是,你可以指证那不是。我若说的是,你为什么打我呢。

亚那就把耶稣解到大祭司该亚法那里,仍是捆着解去的。

西门彼得正站着烤火,有人对他说,你不也是他的门徒吗。彼得不承认,说,我不是。

有大祭司的一个仆人,是彼得削掉耳朵那人的亲属,说,我不是看见你同他在园子里吗。

彼得又不承认。立时鸡就叫了。

众人将耶稣从该亚法那里往衙门内解去。那时天还早。他们自己却不进衙门,恐怕染了污秽,不能吃逾越节的筵席

彼拉多就出来,到他们那里,说,你们告这人是为什么事呢。

他们回答说,这人若不是作恶的,我们就不把他交给你。

彼拉多说,你们带他去,按着你们的律法审问他吧。犹太人说,我们没有杀人的权柄。

这要应验耶稣所说,自己将要怎样死的话了。

彼拉多又进了衙门,叫耶稣来,对他说,你是犹太人的王吗。

耶稣回答说,这话是你自己说的,还是别人论我对你说的呢。

彼拉多说,我岂是犹太人呢。你本国的人和祭司长,把你交给我。你作了什么事呢。

耶稣回答说,我的国不属这世界。我的国若属这世界,我的臣仆必要争战,使我不至于被交给犹太人。只是我的国不属这世界。

彼拉多就对他说,这样,你是王吗。耶稣回答说,你说我是王。我为此而生,也为此来到世间,特为给真理作见证。凡属真理的人,就听我的话。

彼拉多说,真理是什么呢。说了这话,又出来到犹太人那里,对他们说,我查不出他有什么罪来。

但你们有个规矩,在逾越节要我给你们释放一个人,你们要我给你们释放犹太人的王吗。

他们又喊着说,不要这人,要巴拉巴。这巴拉巴是个强盗。

\chapter{第19章}
当下彼拉多将耶稣鞭打了。

兵丁用荆棘编作冠冕,戴在他头上,给他穿上紫袍。

又挨近他说,恭喜犹太人的王阿。他们就用手掌打他。

彼拉多又出来对众人说,我带他出来见你们,叫你们知道我查不出他有什么罪来。

耶稣出来,戴着荆棘冠冕,穿着紫袍。彼拉多对他们说,你们看这个人。

祭司长和差役看见他,就喊着说,钉他十字架,钉他十字架。彼拉多说,你们自己把他钉十字架吧。我查不出他有什么罪来。

犹太人回答说,我们有律法,按那律法,他是该死的,因他以自己为神的儿子。

彼拉多听见这话,越发害怕。

又进衙门,对耶稣说,你是那里来的。耶稣却不回答。

彼拉多说,你不对我说话吗。你岂不知我有权柄释放你,也有权柄把你钉十字架吗。

耶稣回答说,若不是从上头赐给你的,你就毫无权柄辨我。所以把我交给你的那人,罪更重了。

从此彼拉多想要释放耶稣。无奈犹太人喊着说,你若释放这个人,就不是凯撒的忠臣。(原文作朋友)凡以自己为王的,就是背叛凯撒了。

彼拉多听见这话,就带耶稣出来,到了一个地方,名叫铺华石处,希伯来话叫厄巴大,就在那里坐堂。

那日是逾越节的日子,约有午正。彼拉多对犹太人说,看哪,这是你们的王。

他们喊着说,除掉他,除掉他,钉他在十字架上。彼拉多说,我可以把你们的王钉十字架吗。祭司长回答说,除了凯撒,我们没有王。

于是彼拉多将耶稣交给他们去钉十字架。

他们就把耶稣带了去。耶稣背着自己的十字架出来,到了一个地方,名叫髑髅地,希伯来话叫各各他。

他们就在那里钉他在十字架上,还有两个人和他一同钉着,一边一个,耶稣在中间。

彼拉多又用牌子写了一个名号,安在十字架上。写的是犹太人的王,拿撒勒人耶稣。

有许多犹太人念这名号。因为耶稣被钉十字架的地方,与城相近,并且是用希伯来,罗马,希腊,三样文字写的。

犹太人的祭司长,就对彼拉多说,不要写犹太人的王。要写他自己说我是犹太人的王。

彼拉多说,我所写的,我已经写上了。

兵丁既然将耶稣钉在十字架上,就拿他的衣服分为四分,每兵一分。又拿他的里衣。这件里衣,原是没有缝儿,是上下一片织成的。

他们就彼此说,我们不要撕开,只要拈阄,看谁得着。这要应验经上的话说,他们分了我的外衣,为我的里衣拈阄。兵丁果然作了这事。

站在耶稣十字架旁边的,有他母亲,与他母亲的姊妹,并革罗吧的妻子马利亚,和抹大拉的马利亚。

耶稣见母亲和他所爱的那个门徒站在旁边,就对他说,母亲,(原文作妇人)看你的儿子。

又对那门徒说,看你的母亲。从此那门徒就接他到自己家里去了。

这事以后,耶稣知道各样的事已经成了,为要使经上的话应验,就说,我渴了。

有一个器皿盛满了醋,放在那里。他们就拿海绒蘸满了醋,绑在牛膝草上,送到他口。

耶稣尝(原文作受)了那醋,就说,成了。便低下头,将灵魂交付神了。

犹太人因这日是豫备日,又因那安息日是个大日,就求彼拉多叫人打断他们的腿,把他们拿去,免得尸首当安息日留在十字架上。

于是兵丁来,把头一个的腿,并与耶稣同钉第二个人的腿,都打断了。

只是来到耶稣那里,见他已经死了,就不打断他的腿。

惟有一个兵拿枪扎他的肋旁,随既有血和水流出来。

看见这事的那人就作见证,他的见证也是真的,并且他知道自己所说的是真的,叫你们也可以信。

这些事成了,为要应验经上的话说,他的骨头,一根也不可折断。

经上又有一句说,他们要仰望自己所扎的人。

这些事以后,有亚利马太人约瑟,是耶稣的门徒,只因怕犹太人,就暗暗的作门徒,他来求彼拉多,要把耶稣的身体领去。彼拉多允准,他就把耶稣的身体领去了。

又有尼哥底母,就是先前夜里去见耶稣的,带着没药,和沈香,约有一百斤前来。

他们就照犹太人殡葬的规矩,把耶稣的身体,用细麻布加上香料裹好了。

在耶稣钉十字架的地方,有一个园子。园子里有一座新坟墓,是从来没有葬过人的。

只因是犹太人的豫备日,又因那坟墓近,他们就把耶稣安放在那里。

\chapter{第20章}
七日的第一日清早,天还黑的时候,抹大拉的马利亚来到坟墓那里,看见石头从坟墓挪开了。

就跑来见西门彼得,和耶稣所爱的那个门徒,对他们说,有人把主从坟墓里挪了去,我们不知道放在那里。

彼得和那门徒就出来,往坟墓那里去。

两个人同跑,那门徒比彼得跑的更快,先到了坟墓。

低头往里看,就见细麻布还放在那里。只是没有进去。

西门彼得随后也到了,进坟墓里去,就看见细麻布还放在那里。

又看见耶稣的裹头巾,没有和细麻布放在一处,是另一处卷着。

先到坟墓的那门徒也进去,看见就信了。

因为他们还不明白圣经的意思,就是耶稣必要从死里复活。

于是两个门徒回自己的住处去了。

马利亚却站在坟墓外面哭。哭的时候,低头往坟墓里看,

就见两个天使,穿着白衣,在安放耶稣身体的地方坐着,一个在头,一个在脚。

天使对他说,妇人,你为什么哭。他说,因为有人把我主挪了去,我不知道放在那里。

说了这话,就转过身来,看见耶稣站在那里,却不知道是耶稣。

耶稣问他说,妇人,为什么哭,你找谁呢。马利亚以为是看园的,就对他说,先生,若是你把他移了去,请告诉我,你把他放在那里,我便去取他。

耶稣说,马利亚。马利亚就转过来,用希伯来话对他说,拉波尼。(拉波尼就是夫子的意思

耶稣说,不要摸我。因为我还没有升上去见我的父。你往我弟兄那里去,告诉他们说,我要升上去,见我的父,也是你们的父。见我的神,也是你们的神。

抹大拉的马利亚就去告诉门徒说,我已经看见了主。他又将主对他说的这话告诉他们。

那日,(就是七日的头一日)晚上,门徒所在的地方,因怕犹太人,门都关了。耶稣来站在当中,对他们说,愿你们平安。

说了这话,就把手和肋旁,指给他们看。门徒看见主,就喜乐了。

耶稣又对他们说,愿你们平安。父怎样差遣了我,我也照样差遣你们。

说了这话,就向他们吹一口气,说,你们受圣灵。

你们赦免谁的罪,谁的罪就赦免了。你们留下谁的罪,谁的罪就留下了。

那十二个门徒中,有称为抵土马的多马。耶稣来的时候,他没有和他们同在。

那些门徒对他说,我们已经看见主了。多马却说,我非看见他手上的钉痕,用指头探入那钉痕,又用手探入他的肋旁,我总不信。

过了八日,门徒又在屋里,多马也和他们同在,门都关了。耶稣来站在当中说,愿你们平安。

就对多马说,伸过你的指头来,摸(摸原文作看)我的手。伸出你的手来,探入我的肋旁。不要疑惑,总要信。

多马说,我的主,我的神。

耶稣对他说,你因看见了我才信。那没有看见就信的,有福了。

耶稣在门徒面前,另外行了许多神迹,没有记在这书上。

但记这些事,要叫你们信耶稣是基督,是神的儿子。并且叫你们信了他,就可以因他的名得生命。

\chapter{第21章}
这些事以后,耶稣在提比哩亚海边,又向门徒显现。他怎样显现记在下面。

有西门彼得,和称为抵土马的多马,并加利利的迦拿人拿但业,还有西庇太的两个儿子,又有两个门徒,都在一处。

西门彼得对他们说,我打鱼去。他们说,我们也和你同去。他们就出去,上了船,那一夜并没有打着什么。

天将亮的时候,耶稣站在岸上。门徒却不知道是耶稣。

耶稣就对他们说,小子,你们有吃的没有。他们回答说,没有。

耶稣说,你们把网撒在船的右边,就必得着。他们便撒下网去,竟拉不上来了,因为鱼甚多。

耶稣所爱的那门徒对彼得说,是主。那时西门彼得赤着身子,一听见是主,就束上一件外衣,跳在海里。

其馀的门徒(离岸不远,约有二百肘,(古时以肘为尺,一肘约有今时尺半)就在小船把那网鱼拉过来。

他们上了岸,就看见那里有炭火,上面有鱼,又有饼。

耶稣对他们说,把刚才打的鱼,拿几条来。

西门彼得就去,(或作上船)把网拉到岸上,那网满了大鱼,共一百五十三条。鱼虽这样多,网却没有破。

耶稣说,你们来吃早饭。门徒中没有一个敢问他,你是谁,因为知道是主。

耶稣就来拿饼和鱼给他们。

耶稣从死里复活以后,向门徒显现,这是第三次。

他们吃完了早饭,耶稣对西门彼得说,约翰的儿子西门,(约翰马太十六章十七节称约拿)你爱我比这些更深吗。彼得说,主阿,是的。你知道我爱你。耶稣对他说,你喂养我的小羊。

耶稣第二次又对他说,约翰的儿子西门,你爱我吗。彼得说,主阿,是的。你知道我爱你。耶稣说,你牧养我的羊。

第三次对他说,约翰的儿子西门,你爱我吗。彼得因为耶稣第三次对他说,你爱我吗,就忧愁,对耶稣说,主阿,你是无所不知的,你知道我爱你。耶稣说,你喂养我的羊。

我实实在在的告诉你,你年少的时候,自己束上带子,随意往来,但年老的时候,你要伸手来,别人要把你束上,带你到不愿意去的地方。

耶稣说这话,是指着彼得要怎样死荣耀神。说了这话,就对他说,你跟从我吧。

彼得转过来,看见耶稣所爱的那门徒跟着,就是在晚饭的时候,靠着耶稣胸膛,说,主阿,卖你的是谁的那门徒。

彼得看见他,就问耶稣说,主阿,这人将来如何。

耶稣对他说,我若要他等到我来的时候,与你何干。你跟从我吧。

于是这话传在弟兄中间,说那门徒不死。其实耶稣不是说他不死。乃是说我若要他等到我来的时候,与你何干。

为这些事作见证,并且记载这些事的,就是这门徒。我们也知道他的见证是真的。

耶稣所行的事,还有许多,若是一一的都写出来,我想所写的书,就是世界也容不下去了。


\backmatter
\end{document}
